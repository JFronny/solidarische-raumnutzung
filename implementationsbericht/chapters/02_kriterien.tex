%!TEX root = ../main.tex

\chapter{Kriterien}
\label{chap:kriterien}

In diesem Kapitel wird auf die Kriterien eingegangen, die im Pflichtenheft spezifiziert wurden.
Für jedes Kriterium wird angegeben, ob es im Rahmen der Implementierung umgesetzt wurde, beziehungsweise weshalb nicht.
Wurde ein Kriterium abweichend vom Pflichten und Entwurfsheft implementiert werden hier diese Anpassungen erläutert und aufgeführt, weshalb die Änderungen notwendig waren.

\section{Musskriterien}\label{sec:musskriterien}

Folgende Musskriterien wurden im Pflichtenheft aufgeführt:

\must{1}{Die Anwendung muss als Web-Applikation realisiert werden.}
\must{2}{Nutzende der Anwendung müssen sich mit ihrem KIT-Konto per \gls{OIDC} oder einem lokalen Gastkonto anmelden können.}
\must{3}{Nutzende müssen sich abmelden können.}
\must{4}{Die Anwendung muss die Ansichten Kalender, Termin, Termin-erstellen, Login, Kontenliste und Terminübersicht anbieten.}
\must{5}{Die Ansicht \textit{Kalender} muss einen klaren Überblick über die bereits reservierten Zeiten geben.}
\must{6}{Die Ansicht \textit{Kalender} muss die Öffnungszeiten des Raumes darstellen.}
\must{7}{Die Ansicht \textit{Kalender} muss die Termine des/r angemeldeten Nutzenden hervorgehoben darstellen.}
\must{8}{Die Ansicht \textit{Termin} muss die Möglichkeit bieten, genauere Informationen über einen Termin darzustellen.}
\must{9}{Die Ansicht \textit{Termin-Erstellen} muss die Möglichkeit bieten, einen Raum für eine bestimmte Zeitperiode zu reservieren.}
\must{10}{Bei der Reservierung eines Raumes muss optional die Möglichkeit bestehen eine Beschreibung zu hinterlegen. Hierbei müssen die Nutzenden klar darauf hingewiesen werden, wer diese Daten einsehen kann.}
\must{11}{Die Terminübersicht muss den Nutzenden die Möglichkeit bieten, all ihre Termine zu verwalten.}
\must{12}{Die Priorität eines Termins ist in drei Stufen gegliedert.}
\must{13}{Beim Erstellen eines Termins wählen die Nutzenden aus, ob sie ihren Raum mit anderen Personen teilen möchten. Dabei stehen die Optionen \textit{Ja}, \textit{Nein} und \textit{Auf Anfrage} zur Auswahl.}
\must{14}{Die Anwendung muss Terminkonflikte reibungslos mithilfe der Prioritäten lösen können. Dabei überschreiben Termine mit höherer Priorität andere.}
\must{15}{Die Anwendung muss in der Lage sein, Nutzende per E-Mail darüber zu informieren, wenn ihr Termin durch einen Termin mit höherer Priorität überschrieben wurde.}
\must{16}{Nutzende der Anwendung müssen in der Lage sein, eine Reservierung zu stornieren.}
\must{17}{Es muss ein Adminkonto geben, welches per Passwort authentifiziert wird. Nur der Server-Admin darf dieses Passwort ändern können.}
\must{18}{Die Ansicht \textit{Kontoliste} muss den Admins die Möglichkeit bieten, einzelne Konten sowie die Anmeldung per Gastkonto zu deaktivieren.}
\must{19}{Admins müssen in der Ansicht \textit{Termin} Termine löschen können.}
\must{20}{Es muss ein farbcodiertes Banner geben, der den aktuellen Status des Raumes anzeigt.}
\must{21}{Admins müssen in der Ansicht \textit{Kalender} die Öffnungszeiten einstellen können}

\section{Musskriterien Implementierung}\label{sec:musskriterien_implementierung}

Die Funktionalität aller Musskriterien wurde in der Implementierung umgesetzt.
Es gibt allerdings kleine Anpassungen, die im Folgenden beschrieben werden.

\todo{Add more musskrit! abweichungen}
\todo{seperate out the changes!}
\mustchange{1}{Umgesetzt.}
\mustchange{2}{Umgesetzt.}
\mustchange{3}{Umgesetzt.}
\mustchange{4}{Die im Pflichtenheft beschriebenen Ansichten sind vorhanden.}
\mustchange{5}{Die Ansicht \textit{Kalender} gibt mit FullCalendar-Events einen Überblick über die reservierten Zeiten.}
\mustchange{6}{Die Ansicht \textit{Kalender} stellt die Öffnungszeiten des Raumes farblich dar.}
\mustchange{7}{Die Ansicht \textit{Kalender} hebt die Termine des angemeldeten Nutzenden mit einem Badge hervor.}
\mustchange{8}{Die Ansicht \textit{Termin} zeigt sowohl Start- und Endzeitpunkt als auch Beschreibung des Termins.}
\mustchange{9}{Die Ansicht \textit{Termin-Erstellen} ermöglicht es, einen Raum für eine bestimmte Zeitperiode zu reservieren.}
\mustchange{10}{Bei der Reservierung eines Raumes kann optional eine Beschreibung hinterlegt werden. Ein Paragraph informiert über die Sichtbarkeit dieser Beschreibung.}
\mustchange{11}{Die Ansicht \textit{Terminübersicht} zeigt alle Termine des angemeldeten Nutzenden in einer Tabelle an.}
\mustchange{12}{Die Priorität eines Termins ist in drei Stufen gegliedert.}
\mustchange{13}{Beim Erstellen eines Termins wählen die Nutzenden aus, ob sie ihren Raum mit anderen Personen teilen möchten. Dabei stehen die Optionen \textit{Ja}, \textit{Nein} und \textit{Auf Anfrage} zur Auswahl.}
\mustchange{14}{Die Anwendung löst Terminkonflikte mithilfe der Prioritäten. Termine mit höherer Priorität überschreiben andere. Nutzende werden dabei über die Überschreibung informiert.}
\mustchange{15}{Nutzende werden per E-Mail informiert, wenn ihr Termin durch einen Termin mit höherer Priorität überschrieben wurde.}
\mustchange{16}{Nutzende können eine Reservierung stornieren. Dies ist über die Terminansicht mit dem Button \textit{Löschen} möglich.}
\mustchange{17}{Es gibt ein Adminkonto, welches per Passwort authentifiziert wird. Es wird per Docker-Compose festgelegt und kann also nur durch den Server-Admin geändert werden.}
\mustchange{18}{Die Ansicht \textit{Kontoliste} ermöglicht es Admins, einzelne Konten sowie die Anmeldung per Gastkonto zu deaktivieren.}
\mustchange{19}{Admins können in der Ansicht \textit{Termin} Termine löschen.}
\mustchange{20}{Ein farbcodiertes Banner zeigt den aktuellen Status des Raumes an.}
\mustchange{21}{Admins können Öffnungszeiten einstellen, für diese Möglichkeit gibt es allerdings eine neue Ansicht.}

\section{Wunschkriterien}\label{sec:wunschkriterien}

Folgende Wunschkriterien wurden im Pflichtenheft aufgeführt:

\wish{1}{Es könnte die Möglichkeit geben, mehr als einen Raum zur Buchung anzubieten. Dabei könnte eine Raumauswahl vor der Ansicht \textit{Kalender} die verschiedenen Möglichkeiten präsentieren. Existiert nur ein Raum, wird diese Auswahl übersprungen.}
\wish{2}{In der Ansicht \textit{Kalender} könnten Feiertage automatisch eingebunden werden.}
\wish{3}{Admins könnten die Möglichkeit haben, geplante Wartungs- und Sperrzeiten einzurichten.}
\wish{4}{Termine könnten nach der Buchung im \gls{iCal}-Format zum Export angeboten werden.}
\wish{5}{Nutzende könnten in der Ansicht \textit{Termin} die Möglichkeit haben, ihre eigenen Termine zu bearbeiten.}
\wish{6}{Die Ansicht \textit{Kalender} könnte visualisieren, welche Termine bereits in der Vergangenheit liegen und wo der Übergang von der Vergangenheit zur Zukunft liegt.}
\wish{7}{Tooltips könnten Nutzenden erklären, wofür bestimmte Elemente der \gls{UI} verwendet werden.}
\wish{9}{Es könnte einen physischen Panik-Button geben.}
\wish{10}{Die Anwendung könnte in der Lage sein, Nutzende zu informieren, falls ein gewünschter Termin frei wird.}
\wish{11}{Es könnte einen Quick-Checkin-Button geben, welcher eine vorausgefüllte Terminerstellung öffnet. Dieser bietet auch eine Alternative zur Interaktion mit dem Kalender.}
\wish{12}{Es könnte einen Quick-Checkout-Button geben, der vorzeitiges Beenden eines Termines ermöglicht.}
\wish{13}{Admins könnten eine Statistik-Ansicht nutzen.}

\section{Wunschkriterien Implementierung}\label{sec:wunschkriterien_implementierung}
Einige dieser Wunschkriterien wurden ebenfalls umgesetzt.
Im Folgenden wird beschrieben, welche Wunschkriterien umgesetzt wurden und welche nicht.

\todo{Add more wunschkrit!}
\todo{seperate out the changes!}
\wishchange{1}{Es gibt die Möglichkeit, mehr als einen Raum zur Buchung anzubieten. Die zur Verfügung stehenden Räume können über die Ansicht \textit{Räume} konfiguriert werden. Dem Nutzer wird falls nötig eine Raumauswahl vor der Ansicht \textit{Kalender} präsentiert. Existiert nur ein Raum, wird diese Auswahl übersprungen.}
\wishchange{2}{In der Ansicht \textit{Kalender} werden Feiertage automatisch eingebunden.}
\wishchange{3}{Geplante Warn- und Sperrzeiten wurden nicht umgesetzt, da sie nicht als notwendig erachtet wurden.}
\wishchange{4}{Termine können nach der Buchung im \gls{iCal}-Format zum Export angeboten werden.}
\wishchange{5}{Nutzende können in der Ansicht \textit{Termin} die Beschreibung ihrer Termine bearbeiten. Um die Oberfläche einfach zu halten, wurde die Bearbeitung anderer Felder nicht umgesetzt.}
\wishchange{6}{Die Ansicht \textit{Kalender} visualisiert, welche Termine bereits in der Vergangenheit liegen und wo der Übergang von der Vergangenheit zur Zukunft liegt. Dafür wird die aktuelle Zeit mit einem roten Strich markiert.}
\wishchange{7}{Tooltips erklären Nutzenden, wofür die wichtigsten Elemente der \gls{UI} verwendet werden. Im Verlauf der QA-Phase sollen diese basierend auf Nutzerfeedback weiter verbessert werden.}
\wishchange{9}{Ein physischer Panik-Button wurde nicht umgesetzt, da er nicht als notwendig erachtet wurde.}
\wishchange{10}{Die Anwendung informiert Nutzende nicht, wenn ein gewünschter Termin frei wird, da keine überzeugende Kombination aus Use-Case und guter Nutzererfahrung gefunden wurde.}
\wishchange{11}{Ein Quick-Checkin-Button wurde umgesetzt. Der Aktuelle Zeitslot ist bei Nutzung des Buttons bereits vorausgefüllt.}
\wishchange{12}{Ein Quick-Checkout-Button wurde umgesetzt. Der Nutzer kann den aktuellen Termin vorzeitig beenden.}
\wishchange{13}{Admins können eine globale Statistik-Ansicht nutzen um die Buchungen pro Wochentag, Tag des Monats und Monat anzuzeigen.}
