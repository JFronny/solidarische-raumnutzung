%!TEX root = ../main.tex

\chapter{Tests und Coverage}
\label{ch:tests}

Für kontinuierliches Testen und zur Sicherstellung der Qualität des Codes wurden sowohl Unit- als auch Integrationstests geschrieben.
Diese werden nach jedem Push auf das Repository durch GitHub CI automatisiert ausgeführt.
Umgesetzt sind die Tests mit JUnit als Testframework, Mockito für Mocking und werden von Gradle gestartet.
Damit sichergestellt werden kann, dass die Tests in einem Realitätsnahen Umfeld laufen, wird eine normale Postgres-Datenbank verwendet,
welche in einem von Gradle gestarteten Docker-Container läuft.
Außerdem setzen wir JaCoCo ein, um die Abdeckung unseres Quellcodes durch die Tests sicherzustellen.
JaCoCo wird ebenfalls von Gradle gestartet und generiert nach jedem Testdurchlauf einen Report, der in GitHub CI angezeigt wird.
Bei einer zu niedrigen Testabdeckung wird außerdem ein fehlgeschlagener Check in GitHub CI generiert.
Ausgenommen von den Tests ist lediglich der Inhalt des Pakets \textit{config}, da dieser lediglich der Konfiguration von Spring dient
und nicht gut durch Unit- oder Integrationstests geprüft werden kann.