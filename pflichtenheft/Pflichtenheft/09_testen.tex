\chapter{Testfälle und Testeszzenarien}
\label{chap:test}
In diesem Kapitel definieren wir die Testfälle und Testfallszenarien.

\section{Basis-Testfälle}

Jeder Produktfunktion entspricht eineme Basis-Testfall. Die Basis-Testfälle sind uten aufgelistet.


\begin{table}[htbp]


  \centering
%    \caption{Überblick von allen Funktionen.}
  \begin{tabularx}{\textwidth}{ l|X|l }
      \textbf{Nr.} & \textbf{Beschreibung} & \textbf{Funktion} \\ \hline\hline
      ⟨T10⟩ & Landeseite besuchen &\ref{F10}\\
      ⟨T20⟩ & Login &\ref{F20} \\
      ⟨T30⟩ & Reservieren &\ref{F30} \\
      ⟨T40⟩ & Löschen von Terminen durch Administratoren &\ref{F40} \\
      ⟨T50⟩ & Deaktivierung von Gastbenutzern &\ref{F50} \\
      ⟨T60⟩ & Benachrichtigung bei freiem Raum &\ref{F60} \\
      ⟨T70⟩ & Anzeige des Raumstatus &\ref{F70} \\
      ⟨T80⟩ & Stornierung einer Reservierung &\ref{F80} \\
      ⟨T90⟩ & Abmelden &\ref{F30} \\
      ⟨T100⟩& Login des Adminkontos &\ref{F100} \\
      ⟨T120⟩ & Terminkonfliktauflösung &\ref{F130} \\
  \end{tabularx}\label{tab:test_table}
\end{table}

\section{Testfallszenarien}
Die Testfallszenarien ergeben sich als Komposition der Basis-Testfälle.\\ \\
\begin{scenario}{10}{Besuch der Landeseite und Anmeldung/Abmeldung}
  \item[Ziel:] Sicherstellen, dass Nutzende die Landeseite aufrufen und sich erfolgreich anmelden bzw.\ abmelden können.
  \begin{enumerate}
    \item Der Nutzende besucht die Landeseite ⟨T10⟩.
    \item Der Nutzende loggt sich ein ⟨T20⟩.
    \item Der Nutzende meldet sich ab ⟨T90⟩.
  \end{enumerate}
\end{scenario}

\pagebreak

\begin{scenario}{20}{Reservierung eines Raums und Stornierung}
  \item[Ziel:] Überprüfen, ob Nutzende erfolgreich Termine reservieren können.
  \begin{enumerate}
    \item Der Nutzende besucht die Landeseite ⟨T10⟩.
    \item Der Nutzende meldet sich an ⟨T20⟩.
    \item Der Nutzende wählt einen verfügbaren Raum aus und reserviert diesen ⟨T30⟩.
    \item Der Nutzende storniert die Reservierung ⟨T80⟩.
  \end{enumerate}
\end{scenario}

\begin{scenario}{30}{Verwaltung durch das Adminkonto}
  \item[Ziel:] Testen der administrativen Funktionalitäten zum Löschen von Terminen und Deaktivieren von Gastbenutzern.
  \begin{enumerate}
    \item Das Adminkonto besucht die Landeseite ⟨T10⟩.
    \item Das Adminkonto meldet sich an ⟨T100⟩.
    \item Das Adminkonto löscht bestehende Termine ⟨T40⟩.
    \item Das Adminkonto deaktiviert die Anmeldung von Gastkonten ⟨T50⟩.
    \item Das Adminkonto meldet sich ab ⟨T90⟩.
    \item Versuch anmelden mit einem Gastkonto sollte nun fehlschlagen ⟨T20⟩.
  \end{enumerate}
\end{scenario}

\todo{this is just wrong!}
\begin{scenario}{40}{Termin anzeigen}
  \item[Ziel:] Überprüfen, ob ein Termin richtig.
  \begin{enumerate}
    \item Der Nutzende besucht die Landeseite ⟨T10⟩.
    \item Der Nutzende meldet sich an ⟨T20⟩.
    \item Der Nutzende wählt einen Raum aus und überprüft dessen Status ⟨T20⟩.
  \end{enumerate}
\end{scenario}

\begin{scenario}{50}{Terminkonflikt auflösung}
  \item[Ziel:] Überprüfen, ob ein Terminkonflikt richtig aufgelöst wird.
  \begin{enumerate}
    \item Konto 1 meldet sich an ⟨T20⟩.
    \item Konto 1 erstellt einen Termin, und gibt die zu testende Priorität, Raumteilungsoption und Zeitperiode ein ⟨T20⟩.
    \item Konto 1 meldet sich ab und Konto 2 meldet sich an.
    \item Konto 2 erstellt einen Termin, welcher den anderen überlappt.
    \item Es wird überprüft, ob die erwartete Konfliktauflösung stattgefunden hat und ggf.\ die dafür benötigten E-Mails versendet wurden ⟨T130⟩.
  \end{enumerate}
\end{scenario}







