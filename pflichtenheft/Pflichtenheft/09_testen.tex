\chapter{Testfälle und Testszenarien}
\label{chap:test}
In diesem Kapitel definieren wir die Testfälle und Testfallszenarien.

\section{Basis-Testfälle}

Jede Produktfunktion entspricht einem Basis-Testfall. Die Basis-Testfälle sind unten aufgelistet.


\begin{table}[htbp]


  \centering
%    \caption{Überblick von allen Funktionen.}
  \begin{tabularx}{\textwidth}{ l|X|l }
      \textbf{Nr.} & \textbf{Beschreibung} & \textbf{Funktion} \\ \hline\hline
      ⟨T10⟩ & Landeseite besuchen &\ref{F10}\\
      ⟨T20⟩ & Login &\ref{F20} \\
      ⟨T100⟩& Login des Adminkontos &\ref{F100} \\
      ⟨T30⟩ & Reservieren &\ref{F40} \\
      ⟨T40⟩ & Löschen von Terminen durch Administratoren &\ref{F50} \\
      ⟨T50⟩ & Deaktivierung eines Gastkontos &\ref{F60} \\
      ⟨T60⟩ & Benachrichtigung bei freiem Raum &\ref{F70} \\
      ⟨T70⟩ & Anzeige des Raumstatus &\ref{F80} \\
      ⟨T80⟩ & Stornierung einer Reservierung &\ref{F90} \\
      ⟨T90⟩ & Abmelden &\ref{F30} \\
      ⟨T110⟩& Deaktivierung der Gastfunktion &\ref{F110} \\
      ⟨T120⟩& Öffnungszeiten einstellen &\ref{F120} \\
      ⟨T130⟩& Terminkonfliktauflösung &\ref{F130} \\
  \end{tabularx}\label{tab:test_table}
\end{table}

\section{Testfallszenarien}
Die Testfallszenarien ergeben sich als Komposition der Basis-Testfälle.\\ \\
\begin{scenario}{10}{Besuch der Landeseite und Anmeldung/Abmeldung}
  \item[Ziel:] Sicherstellen, dass Nutzende die Landeseite aufrufen und sich erfolgreich anmelden bzw. abmelden können.
  \begin{enumerate}
    \item Nutzer*in besucht die Landeseite. ⟨T10⟩
    \item Nutzer*in loggt sich ein. ⟨T20⟩
    \item Nutzer*in meldet sich ab. ⟨T90⟩
  \end{enumerate}
\end{scenario}

\pagebreak

\begin{scenario}{20}{Reservierung eines Raums und Stornierung}
  \item[Ziel:] Überprüfen, ob Nutzende erfolgreich Termine reservieren können.
  \begin{enumerate}
    \item Nutzer*in besucht die Landeseite ⟨T10⟩
    \item Nutzer*in meldet sich an ⟨T20⟩
    \item Nutzer*in wählt einen verfügbaren Raum aus und reserviert diesen ⟨T30⟩
    \item Nutzer*in storniert die Reservierung ⟨T80⟩
  \end{enumerate}
\end{scenario}

\begin{scenario}{30}{Verwaltung durch das Adminkonto}
  \item[Ziel:] Testen der administrativen Funktionalitäten zum Löschen von Terminen und Deaktivieren von Gastbenutzern.
  \begin{enumerate}
    \item Administrator*in besucht die Landeseite ⟨T10⟩.
    \item Administrator*in meldet sich an ⟨T100⟩.
    \item Administrator*in löscht bestehende Termine ⟨T40⟩.
    \item Administrator*in deaktiviert die Anmeldung von Gastkonten ⟨T110⟩.
    \item Administrator*in meldet sich ab ⟨T90⟩.
    \item Versuch sich mit einem Gastkonto anzumelden sollte nun fehlschlagen ⟨T20⟩.
  \end{enumerate}
\end{scenario}

\todo{this is just wrong!}
\begin{scenario}{40}{Termin anzeigen}
  \item[Ziel:] Überprüfen, ob ein Termin richtig.
  \begin{enumerate}
    \item Nutzer*in besucht die Landeseite ⟨T10⟩.
    \item Nutzer*in meldet sich an ⟨T20⟩.
    \item Nutzer*in wählt einen Raum aus und überprüft dessen Status ⟨T20⟩.
  \end{enumerate}
\end{scenario}







