%!TEX root = ../Pflichtenheft.tex

% Kapitel 4
%-------------------------------------------------------------------------------

\chapter{Produktfunktionen}
\label{chap:product_functions}

\iffalse
In Abhängigkeit von den gewählten Konzepten erfolgt hier eine Konkretisierung
und Detaillierung der Funktionen aus den Use-Case-Diagrammen und ggf. dem
Angebot.

Die Produktfunktionen müssen die Kriterien aus den Zielbestimmungen abdecken.
Dabei kann es je nach Kriterium eine oder mehrere Funktion geben.
Das nachfolgende Format sollte für einige Funktionsbeschreibungen übernommen werden:

Beispiel:\\
\begin{function}{10}{Lagerverwaltung}
    \item[Anwendungsfall:] Automatisches Einlagern
    \item[Anforderung:] \lfk{20} (Wenn kein Lastenheft vorhanden, dann Kriterium aus den Zielbestimmungen, z.B.: \ref{RM1})
    \item[Ziel:] Ein Reifen erscheint am Systemeingang (Scanner), erhält einen Lagerplatz
    zugewiesen und wird dort eingelagert.
    \item[Vorbedingung:] Das Scannen des Barcode-Reifens muss erfolgreich sein, sonst kann
    der Typ nicht ermittelt werden. Solche unbekannten Reifen werden direkt in den
    Überlauf gefördert.
    \item[Nachbedingung Erfolg:] Reifen ist physikalisch eingelagert und logisch in der
    Datenbank verbucht.
    \item[Nachbedingung Fehlschlag:] Der Reifen wurde infolge gestörter Fördermechanik
    nicht eingelagert (liegt im Überlauf) oder produzierte aufgrund inkonsistenter
    Datenbank einen "`Platz belegt"' - Fehler beim Anfahren eines irrtümlich
    als frei angenommenen Platzes.
    \item[Akteure:] ~Produktion
    \item[Auslösendes Ereignis:] SPS meldet der Steuerung, dass am Eingangsscanner ein
    Reifen mit Seriennummer X des Typs Y eingetroffen ist.
    \item[Beschreibung:] ~
    \begin{enumerate}
      \item Reifentypinformationen ermitteln (besonders Höhe des Reifens bei Wahl zwischen unterschiedlich hohen Lagerplätzen wichtig).
      \item Alle Module ermitteln, die\\
    -  Platz auf den Einlagerstichen haben\\
    -  momentan nicht im Störungszustand sind\\
    -  freie Lagerplätze in der geforderten Höhe aufweisen.
      \item Lagerplatz nach Gleichverteilungsgrundsatz bestimmen.
      \item Reifen auf den Einlagerstich des gewählten Moduls befördern.
      \item Sobald er auf dem vordersten Platz des Einlagerstichs steht, dem Modul den Befehl zur Reifenaufnahme und Einlagerung auf den gewählten Platz schicken.
    \end{enumerate}
    \item[Erweiterung:] (optional)\\
        2a Zur Effizienzsteigerung auch Module ansteuern, die momentan keinen Platz auf
        den Einlagerstichen haben, aber wahrscheinlich so schnell einlagern, dass der
        Reifen nach der Fahrtzeit zum Modul auf den Stich eingelagert werden kann
        (Überwachung des "`Unterwegsbestandes"' an Reifen für ein bestimmtes
        Modul).\\
        3a Lagerplatz des Reifens möglichst nah zum Einlagerstich im RBG wählen	(kürzere RBG-Fahrtzeiten).
    \item[Alternativen:] (optional)\\
        2a Wenn kein Lagerplatz gefunden wird, Reifen zum Überlauf schicken (der
        Einlagerförderer wird niemals angehalten!).
\end{function}
\fi

\begin{function}{10}{Login}
    \item[Anwendungsfall:] Login
    \item[Anforderung:]~\ref{RM2}
    \item[Ziel:] Der Benutzer kann sich mit seinem KIT-Konto oder einem lokalen Gastkonto anmelden.
    \item[Vorbedingung:] Der Benutzer hat eine Internetverbindung und die Anwendung ist gestartet.
    \item[Nachbedingung Erfolg:] Der Benutzer ist angemeldet und kann Ereignisse ansehen und bearbeiten.
    \item[Nachbedingung Fehlschlag:] Der Benutzer ist nicht angemeldet und kann keine Ereignisse ansehen oder bearbeiten.
    \item[Auslösendes Ereignis:] Der Benutzer öffnet die Anwendung und versucht, mit einem Ereignis zu interagieren.
    \item[Beschreibung:] ~
    \begin{enumerate}
        \item Der Benutzer wählt in einem Dialog die Anmeldemethode aus.
        \item Der Benutzer wird, falls nötig, auf die KIT-Login-Seite weitergeleitet.
        \item Falls keine lokale Anmeldung gewählt wurde, gibt der Benutzer seine Zugangsdaten ein.
        \item Die Anwendung überprüft die Zugangsdaten und meldet den Benutzer an.
        \item Der Benutzer wird zur nächsten Seite weitergeleitet.
    \end{enumerate}
\end{function}

\begin{function}{20}{Reservierung}
    \item[Anwendungsfall:] Reservierung
    \item[Anforderung:] \ref{RM3} und \ref{RM5}
    \item[Ziel:] Der Benutzer kann einen Raum reservieren.
    \item[Vorbedingung:] Der Benutzer ist angemeldet und hat eine Internetverbindung.
    \item[Nachbedingung Erfolg:] Eine Reservierung wird gespeichert und der Benutzer wird zur Buchungsseite weitergeleitet.
    \item[Nachbedingung Fehlschlag:] Die Reservierung wird nicht gespeichert und der Benutzer erhält eine Fehlermeldung.
    \item[Auslösendes Ereignis:] Der Benutzer wählt einen Raum und einen Startzeitpunkt.
    \item[Beschreibung:] ~
    \begin{enumerate}
        \item Der Benutzer prüft einen Start- und Endzeitpunkt und kann sie gegebenenfalls anpassen.
        \item Der Benutzer gibt optional eine Beschreibung an.
        \item Der Benutzer gibt an, ob er bereit ist, die Reservierung zu teilen.
        \item Der Benutzer gibt optional eine E-Mail-Adresse an, um über Änderungen informiert zu werden.
        \item Falls eine E-Mail-Adresse angegeben wurde, kann der Benutzer diese als in der Oberfläche sichtbar oder unsichtbar markieren.
        \item Der Benutzer bestätigt seine Eingaben.
        \item Die Anwendung prüft, ob der Raum in dem Zeitraum verfügbar ist.
        \item Falls ja, wird die Reservierung gespeichert und der Nutzer wird weitergeleitet.
              Wenn nötig, wird eine E-Mail an die Ersteller überschriebener Reservierungen gesendet.
    \end{enumerate}
\end{function}