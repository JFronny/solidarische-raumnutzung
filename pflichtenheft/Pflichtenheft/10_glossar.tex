\newglossaryentry{CI}{name=CI/CD, description={Baut automatisiert Software, führt Tests durch und veröffentlicht Artefakte}}
\newglossaryentry{Docker}{name=Docker, description={Software zur Bereitstellung von Anwendungen innerhalb von Containern}}
\newglossaryentry{Container}{name=Container, description={Isolierte Umgebung um Software unabhängig von der zugrunde liegenden Umgebung auszuführen}}
\newglossaryentry{Devcontainer}{name=Devcontainer, description={Container, welcher eine Entwicklungsumgebung bereitstellt}}
\newglossaryentry{Browser}{name=Browser, description={Software zum Navigieren von Webseiten, zum Beispiel Firefox oder Chrome}}
\newglossaryentry{Git}{name=Git, description={Software zur Versionsverwaltung von Softwareprojekten}}
\newglossaryentry{GitHub}{name=GitHub, description={Plattform zur Versionsverwaltung von Softwareprojekten, nutzt Git}}
\newglossaryentry{IDE}{name=IDE, description={(Integrated Development Environment) Software, welche alle Werkzeuge zur Softwareentwicklung in einem Programm kombiniert}}
\newglossaryentry{AMD64}{name=AMD64, description={Verbreitete Prozessorarchitektur von Intel und AMD}}
\newglossaryentry{RAM}{name=RAM, description={\ (Random Access Memory) Arbeitsspeicher}}
\newglossaryentry{VM}{name=VM, description={\ (Virtuelle Maschine) Software zur Simulation eines Computers}}
\newglossaryentry{PostgreSQL}{name=PostgreSQL, description={Objekt-Relationales Datenbankmanagementsystem, welches zum Speichern und Verwalten von Daten verwendet wird}}
\newglossaryentry{HTML}{name=HTML, description={\ (Hypertext Markup Language) Eine Sprache, um die Struktur und den Inhalt einer Website zu definieren}}
\newglossaryentry{CSS}{name=CSS, description={Cascading Style Sheets, ist eine Sprache um das Visuelle aussehen einer Website zu definieren}}
\newglossaryentry{JavaScript}{name=JavaScript, description={Programmiersprache um das Logische verhalten von Webseiten zu steuern}}
\newglossaryentry{SSR}{name=SSR, description={(Server-Side Rendering) Methode, um das HTML einer Website auf dem Server zu produzieren, statt im Browser}}
\newglossaryentry{REST}{name=REST, description={\ (Representational State Transfer) Architekturstil von APIs für das Internet}}
\newglossaryentry{API}{name=API, description={Schnittstelle auf Quelltext-Ebene um anderen Programmen funktionen zur Verfügung zu stellen}}
\newglossaryentry{UI}{name=UI, description={(User Interface) Typischerweise visuelle Oberfläche, mit welcher der Nutzer interagiert}}
\newglossaryentry{OIDC}{name=OIDC, description={\ (OpenID Connect) Authorisierungsframework welches vom KIT genutzt wird um dritten Webseiten Logins basierend auf KIT-Konten bereitzustellen}}
\newglossaryentry{Gradle}{name=Gradle, description={Build-Management-Tool welches auf Java basiert}}
\newglossaryentry{iCal}{name=iCal, description={Dateiformat zur Speicherung von Kalenderdaten}}
\newglossaryentry{WCAG}{name=WCAG, description={\ (Web Content Accessibility Guidelines) Standard zur Barrierefreien gestaltung von Webseiten.}}