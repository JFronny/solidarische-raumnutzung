%!TEX root = ../Pflichtenheft.tex

\chapter{Produktdaten}
\label{chap:product_data}

\iffalse
Die langfristig zu speichernden Daten sind aus Benutzersicht detaillierter zu beschreiben.
Dabei bietet sich eine formale Beschreibung an, um eine größere Präzisierung zu erreichen.
Es sollte eine Menge an erwarteten Daten angegeben werden.

Es kann die Darstellung gemäß Beispiel verwendet werden (alternativ kann auch ein Klassendiagramm mit entsprechender Beschreibung erstellt werden):\\

\begin{data}{10}{Lagerdaten}
	Daten der Lagerplätze (max. 5.000):\\
	-  Modulnummer,\\
	-  Regalseite,\\
	-  Regalspalte,\\
	-  Regalzeile,\\
	-  Fachhöhe,\\
	-  Platzsperre (0 = nicht gesperrt, 1 = gesperrt für Einlagerung, 2 = gesperrt
	   für Auslagerung, 3 = gesperrt für alle Zugriffe),\\
	-  Reifenstatus (0 = frei,1 = reserviert für Einlagerung, 2= belegt, 3 =
	   reserviert für Auslagerung),\\
	-  Reifenseriennummer.\\
\end{data}

\begin{data}{20}{Moduldaten}
	Daten der Module (max. 20):\\
	-  Modulnummer,\\
	-  Sperrkennzeichen (0 = nicht gesperrt, 1 = gesperrt für Einlagerung, 2 =
	   gesperrt für Auslagerung, 3 = gesperrt für alle Zugriffe),\\
	-  maximale Kapazität,\\
	-  freie Kapazität,\\
	-  belegte Plätze (ergibt sich aus Status und Zahl der zugeordneten
	   Lagerplätze, wird aus Geschwindigkeitsgründen allerdings redundant
	   mitgeführt).
\end{data}
\fi

Die Anwendung verwendet den Server als zentralen Speicherort für alle Daten.
Die Daten werden in einer \gls{PostgreSQL}-Datenbank gespeichert.
Auf dem Client werden nur temporäre Daten gespeichert, die für die Funktionalität der Anwendung notwendig sind.


\subsection*{Clientdaten}
\begin{itemize}
    \item Anmeldungscookie (falls der Benutzer anonym angemeldet ist).
          Diese laufen nach 30 Tagen ab.
    \item Zustand der Anwendung (z.B.\ aktuelle Seite, geöffnete Dialoge)
\end{itemize}

\subsection*{Serverdaten}
\begin{itemize}
    \item Benutzerdaten (dauerhaft gespeichert)
    \begin{itemize}
        \item Benutzername (oder \textit{Anonym} für anonyme Benutzer)
        \item (optional) E-Mail-Adresse
        \item OAuth- oder Cookie-Token
        \item Blockierungsstatus (wahr oder falsch)
    \end{itemize}
    \item Ereignisdaten (30 Tage nach dem Termin werden diese Daten anonymisiert)
    \begin{itemize}
        \item Start- und Endzeitpunkt (Datenbank native Zeitdarstellung)
        \item Beschreibung
        \item Raum
        \item Ersteller
        \item Priorität
        \item Kollaborationsmodus
        \item (optional) E-Mail-Adresse
        \item Sichtbarkeit der E-Mail-Adresse
    \end{itemize}
    \item Raumdaten (dauerhaft gespeichert)
    \begin{itemize}
        \item Raumname
        \item Raumbeschreibung
        \item Raumbild
    \end{itemize}
\end{itemize}