%!TEX root = ../Pflichtenheft.tex

\chapter{Produktdaten}
\label{chap:product_data}

Die langfristig zu speichernden Daten sind aus Benutzersicht detaillierter zu
beschreiben. Dabei bietet sich eine formale Beschreibung an, um eine größere Präzisierung zu erreichen.
Es sollte eine Menge an erwarteten Daten angegeben werden.

Es kann die Darstellung gemäß Beispiel verwendet werden (alternativ kann auch ein Klassendiagramm mit entsprechender Beschreibung erstellt werden):\\

\begin{data}{10}{Lagerdaten}
	Daten der Lagerplätze (max. 5.000):\\
	-  Modulnummer,\\
	-  Regalseite,\\
	-  Regalspalte,\\
	-  Regalzeile,\\
	-  Fachhöhe,\\
	-  Platzsperre (0 = nicht gesperrt, 1 = gesperrt für Einlagerung, 2 = gesperrt
	   für Auslagerung, 3 = gesperrt für alle Zugriffe),\\
	-  Reifenstatus (0 = frei,1 = reserviert für Einlagerung, 2= belegt, 3 =
	   reserviert für Auslagerung),\\
	-  Reifenseriennummer.\\
\end{data}

\begin{data}{20}{Moduldaten}
	Daten der Module (max. 20):\\
	-  Modulnummer,\\
	-  Sperrkennzeichen (0 = nicht gesperrt, 1 = gesperrt für Einlagerung, 2 =
	   gesperrt für Auslagerung, 3 = gesperrt für alle Zugriffe),\\
	-  maximale Kapazität,\\
	-  freie Kapazität,\\
	-  belegte Plätze (ergibt sich aus Status und Zahl der zugeordneten
	   Lagerplätze, wird aus Geschwindigkeitsgründen allerdings redundant
	   mitgeführt).
\end{data}
