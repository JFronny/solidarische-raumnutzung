%!TEX root = ../Pflichtenheft.tex

% Kapitel 6
%-------------------------------------------------------------------------------

\chapter{Nichtfunktionale Anforderungen}
\label{chap:non_functional_req}

In diesem Kapitel wird festgelegt, welche Qualitätsmerkmale das zu entwickelnde
Produkt in welcher Qualitätsstufe besitzen soll. Anschließend werden die als am
wichtigsten bezeichneten Qualitätsmerkmale operationalisiert, d.h. in konkrete
Produktanforderungen detailliert, falls sie nicht als allgemeine Richtlinie (z.
B. Standard, Norm) zur Verfügung gestellt werden können.


Die oben als am wichtigsten bezeichneten Qualitätsmerkmale werden im Folgenden
operationalisiert, d.h. in konkrete Produktanforderungen detailliert oder es
wird angegeben, welche Richtlinie (z. B. Standard, Norm) einzuhalten ist. Diese
Qualitätsanforderungen werden wie im Beispiel definiert. Zu prüfen ist, ob die
gewünschte Qualität mit den in Produktdaten genannten Datenmengen erreicht
werden kann.


Beispiele:

\notfunctional{1}{Die Anwendung soll schnell und einfach bedienbar sein.}
\notfunctional{2}{Die Anwendung soll auf mobilen sowie Desktop-Endgeräten angepasst laufen.}
\unsure{Kann man hier noch weiter spezifizieren?}
\notfunctional{2}{Die Anwendung soll in allen gängigen Browsern lauffähig sein. Insbesondere beinhaltet dies Firefox und Chrome.}
\notfunctional{5}{Es muss die Anonymität der Nutzer gewährleistet sein.}
\notfunctional{4}{Es soll entsprechend auf Acessibility geachtet werden.}
