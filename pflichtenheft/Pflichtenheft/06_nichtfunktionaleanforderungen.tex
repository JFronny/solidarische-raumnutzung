%!TEX root = ../Pflichtenheft.tex

% Kapitel 6
%-------------------------------------------------------------------------------

\chapter{Nichtfunktionale Anforderungen}
\label{chap:non_functional_req}

\iffalse
In diesem Kapitel wird festgelegt, welche Qualitätsmerkmale das zu entwickelnde
Produkt in welcher Qualitätsstufe besitzen soll.
Anschließend werden die als am wichtigsten bezeichneten Qualitätsmerkmale operationalisiert, d.h.\ in konkrete Produktanforderungen detailliert, falls sie nicht als allgemeine Richtlinie (z.
B.\ Standard, Norm) zur Verfügung gestellt werden können.


Die oben als am wichtigsten bezeichneten Qualitätsmerkmale werden im Folgenden operationalisiert, d.h.\ in konkrete Produktanforderungen detailliert oder es wird angegeben, welche Richtlinie (z.B.\ Standard, Norm) einzuhalten ist.
Diese Qualitätsanforderungen werden wie im Beispiel definiert.
Zu prüfen ist, ob die gewünschte Qualität mit den in Produktdaten genannten Datenmengen erreicht werden kann.
\fi


\notfunctional{1}{Die Anwendung soll schnell und einfach bedienbar sein.}
\notfunctional{2}{Die Anwendung soll auf mobilen sowie Desktop-Endgeräten ohne Einschränkungen nutzbar sein.}
\notfunctional{3}{Die Anwendung soll in allen gängigen \gls{Browser}n lauffähig sein. Insbesondere beinhaltet dies Firefox und Chrome auf Desktop- und Android-Geräten und Safari auf iOS.}
\notfunctional{4}{Die Anonymität der Nutzer soll so weit wie möglich gewährleistet werden.}
\notfunctional{5}{Für Nutzer mit Sehschwäche soll die Anwendung durch Kompatibilität mit Screenreadern und kontrastreiche Farbgebung gut bedienbar sein.}
\notfunctional{6}{Die Anwendung soll in verschiedenen Sprachen dargestellt werden können, insbesondere Deutsch und Englisch.}
\notfunctional{7}{Die Anwendung soll auf mobilen sowie Desktop-Endgeräten ohne Einschränkungen nutzbar sein.}
