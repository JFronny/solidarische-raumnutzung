%!TEX root = ../Pflichtenheft.tex

% Kapitel 6
%-------------------------------------------------------------------------------

\chapter{Nichtfunktionale Anforderungen}
\label{chap:non_functional_req}

In diesem Kapitel wird festgelegt, welche Qualitätsmerkmale das zu entwickelnde
Produkt in welcher Qualitätsstufe besitzen soll. Anschließend werden die als am
wichtigsten bezeichneten Qualitätsmerkmale operationalisiert, d.h. in konkrete
Produktanforderungen detailliert, falls sie nicht als allgemeine Richtlinie (z.
B. Standard, Norm) zur Verfügung gestellt werden können.


Die oben als am wichtigsten bezeichneten Qualitätsmerkmale werden im Folgenden
operationalisiert, d.h. in konkrete Produktanforderungen detailliert oder es
wird angegeben, welche Richtlinie (z. B. Standard, Norm) einzuhalten ist. Diese
Qualitätsanforderungen werden wie im Beispiel definiert. Zu prüfen ist, ob die
gewünschte Qualität mit den in Produktdaten genannten Datenmengen erreicht
werden kann.


Beispiele:

\begin{itemize}

\item  \qualityReq{10}{Die Funktion \ref{F20} darf nicht länger als 5 Sekunden Antwortzeit benötigen.}
\item  \qualityReq{20}{Alle Reaktionszeiten auf Benutzeraktionen müssen unter 2 Sekunden
liegen (außer Funktion \ref{F20}).}
\item  \qualityReq{30}{Die im Rahmen der automatischen Einlagerung \ref{F10} notwendige
Platzwahl für einen am Anmeldescanner gemeldeten Reifen darf aus Gründen der
Kommunikation mit der SPS nicht länger als 3 Sekunden dauern, ansonsten kann
die SPS die Lieferung des Reifens zum richtigen Modul nicht garantieren.}
\item  \qualityReq{40}{Das Produkt soll plattformunabhängig sein.}
\item  \qualityReq{50}{Das Produkt muss anwenderfreundlich sein. (Intuitive Bedienbarkeit
für Benutzer ohne EDV-Vorkenntnisse, umfangreiche Hilfefunktion)}
\item  \qualityReq{60}{Die Produkt soll fehlertolerant bezüglich Bedien- und
Eingabefehler sein.}

\end{itemize}
