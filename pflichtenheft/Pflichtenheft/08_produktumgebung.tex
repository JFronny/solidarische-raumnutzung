%!TEX root = ../Pflichtenheft.tex

% Kapitel 10
% Die Unterkapitel können auch in separaten Dateien stehen,
% die dann mit dem \include-Befehl eingebunden werden.
%-------------------------------------------------------------------------------

\chapter{Technische Produktumgebung}
\label{chap:tech_env}

\section{Entwicklungsumgebung}

\begin{itemize}
    \item Devcontainer für reproduzierbare Entwicklungsumgebung
    \item IntelliJ IDEA als \gls{IDE} und LaTex-Editor
    \item pdfLaTeX zur Kompilierung des Dokuments
    \item \gls{Git} zur Versionskontrolle
    \item \gls{Gradle} zur Build-Automatisierung
    \item \gls{GitHub} Actions für \gls{CI} (inkl. Tests und Dokumentation)
    \item \gls{GitHub} Packages zur Speicherung generierter \gls{Docker}-Images
\end{itemize}

\section{Hardware}

\subsection{Backend}

\begin{itemize}
    \item \gls{AMD64}-kompatibler Prozessor (emuliert in einer \gls{VM})
    \item mindestens 1 GB \gls{RAM}
\end{itemize}

\subsection{Frontend}

\begin{itemize}
    \item Beliebiges (mobiles) Gerät mit Internetzugang
\end{itemize}

\section{Software}

\subsection{Backend}

\begin{itemize}
    \item Ubuntu Linux 24.04 LTS als Betriebssystem
    \item \gls{Docker} zur \gls{Container}isierung der Anwendung
    \item \gls{PostgreSQL} als Datenbank (in einem eigenen \gls{Container})
    \item Applikations-\gls{Container} (gebaut in \gls{CI})
    \item Spring Boot mit Java 21 als Backend-Framework
    \item JTE zur Generierung von \gls{HTML}-Seiten
    \unsure{Wie machen wir HTTPS? Mit Spring oder per NGINX/Caddy? Wie kriegen wir Domäne, Zertifikat, ...?}
\end{itemize}

\subsection{Frontend}

\begin{itemize}
    \item Pure \gls{HTML} mit \gls{JavaScript} zur Interaktivität um maximale Kompatibilität zu gewährleisten
    \item DaisyUI als \gls{CSS}-Framework
    \item FullCalendar für die Kalenderansicht
    \item Moderner \gls{Browser} (Chrome, Firefox, Safari, Edge) zur vollen Funktionalität
\end{itemize}

\section{Schnittstellen}

\begin{itemize}
    \item \gls{SSR} generiert (hauptsächlich) statische \gls{HTML}-Seiten
    \item \gls{HTML}-Forms zur Interaktion mit dem Backend
    \item Minimale \gls{REST}-\gls{API} zum Prüfen auf Verfügbarkeit
    \unsure{Machen wir das?}
\end{itemize}