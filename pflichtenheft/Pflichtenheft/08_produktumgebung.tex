%!TEX root = ../Pflichtenheft.tex

% Kapitel 10
% Die Unterkapitel können auch in separaten Dateien stehen,
% die dann mit dem \include-Befehl eingebunden werden.
%-------------------------------------------------------------------------------

\chapter{Technische Produktumgebung}
\label{chap:tech_env}

\section{Entwicklungsumgebung}

\begin{itemize}
    \item Devcontainer für reproduzierbare Entwicklungsumgebung
    \item IntelliJ IDEA als IDE und LaTex-Editor
    \item pdfLaTeX zur Kompilierung des Dokuments
    \item Git zur Versionskontrolle
    \item Gradle zur Build-Automatisierung
    \item GitHub Actions für CI/CD (inkl. Tests und Dokumentation)
    \item GitHub Packages zur Speicherung generierter Docker-Images
\end{itemize}

\section{Hardware}

\subsection{Backend}

\begin{itemize}
    \item AMD64-kompatibler Prozessor (emuliert in einer VM)
    \item mindestens 1 GB RAM
\end{itemize}

\subsection{Frontend}

\begin{itemize}
    \item Beliebiges (mobiles) Gerät mit Internetzugang
\end{itemize}

\section{Software}

\subsection{Backend}

\begin{itemize}
    \item Ubuntu Linux 24.04 LTS als Betriebssystem
    \item Docker zur Containerisierung der Anwendung
    \item PostgreSQL als Datenbank (in einem eigenen Container)
    \item Applikations-Container (gebaut in CI)
    \item Spring Boot mit Java 21 als Backend-Framework
    \item JTE zur Generierung von HTML-Seiten
    \todo[inline]{Wie machen wir HTTPS? Mit Spring oder per NGINX/Caddy? Wie kriegen wir Domäne, Zertifikat, ...?}
\end{itemize}

\subsection{Frontend}

\begin{itemize}
    \item Pure HTML mit JavaScript zur Interaktivität um maximale Kompatibilität zu gewährleisten
    \item DaisyUI als CSS-Framework
    \item FullCalendar für die Kalenderansicht
    \item Moderner Browser (Chrome, Firefox, Safari, Edge) zur vollen Funktionalität
\end{itemize}

\section{Schnittstellen}

\begin{itemize}
    \item SSR generiert (hauptsächlich) statische HTML-Seiten
    \item HTML-Forms zur Interaktion mit dem Backend
    \item Minimale REST-API zum Prüfen auf Verfügbarkeit
    \todo[inline]{Machen wir das?}
\end{itemize}