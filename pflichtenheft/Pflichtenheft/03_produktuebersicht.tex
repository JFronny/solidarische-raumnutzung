%!TEX root = ../Pflichtenheft.tex

\chapter{Produktübersicht}
\label{chap:product_overview}

\section{Betriebsbedingungen}
\begin{itemize}
    \item Die Software soll in einem \gls{Docker}-\gls{Container} laufen.
    \item Die Software soll weitgehend ohne Unterbrechung laufen.
    \item Ständige Beobachtung ist nicht vorgesehen, die Software soll weitgehend autonom laufen.
    \item Übliche Administrationsaufgaben sollen über die Administrationsoberfläche und ohne Neustart der Software möglich sein.
    \item Die \gls{RAM}-Nutzung soll möglichst gering sein (Ideal: unter 1GB)
\end{itemize}

\section{Use-Cases}

\todo[inline]{Are we still adding usecases?}
\todo[inline]{remove template text when done.}

Der folgende Abschnitt hat die Aufgabe, die Funktionalität des zu entwickelnden Systems grafisch mithilfe von Use-Case-Diagrammen und einer kurzen verbalen
Beschreibung zu charakterisieren.
Im Erklärungstext sollte darauf eingegangen werden, welcher Use-Case welches Kriterium erfüllt.
Es sind so viele Use-Case-Diagramme einzufügen, wie zur vollständigen und übersichtlichen Beschreibung der Systemfunktionalität notwendig sind.

Anmerkungen zu Use-Cases:

\begin{itemize}
	\item Der Zweck eines Use-Cases ist es, zu beschreiben wie ein Aktor das System benutzen kann, um ein bestimmtes Ziel zu erreichen.
	\item Ein Use-Case ist nicht immer eine Funktion, sondern das Ziel des Systems und kann mehrere Funktionen umfassen.
		In diesem Fall kann man dann die einzelnen Funktionen in Aktivitätsdiagrammen beschreiben.
	\item Use-Cases werden klassischerweise so benannt, wie die Ziele aus Sicht der Akteure heißen.
	\item Das Diagramm soll nur den groben Aufbau des Systems beschreiben, damit man sehen kann, ob das System das Richtige tut oder nicht.
\end{itemize}


Beispiel:

\begin{figure}[ht]
\centering
\caption{Use-Case-Diagramm Buchungssystem}
\end{figure}

Dieses Diagramm zeigt ein minimales Use-Case-Diagramm für ein Buchungssystem.
Der Akteur, hier ein Benutzer, der einen Flug buchen möchte, kann einen Flug buchen oder sein Benutzerkonto ansehen.
Diese beiden Use-Cases sind für den Benutzer zielführend.
Ein Use-Case \emph{Flug suchen} wäre hier falsch, denn das ist nicht das Ziel eines Buchungssystems. 
Der Use-Case \emph{Flug buchen} beinhaltet mehrere Funktionen des zugehörigen Systems. 
Diese wären \emph{Einloggen}, \emph{Flug reservieren} usw.\ und sollen im nächsten Kapitel beschrieben werden.
In diesem Fall wäre nun ein Aktivitätsdiagramm zur Beschreibung der einzelnen Funktionen notwendig.


Interessante Use-Cases können mit einem Aktivitätsdiagramm genauer erläutert werden.

\section{Interaktions-Diagramme}
\todo[inline]{Add interations-diagramme}