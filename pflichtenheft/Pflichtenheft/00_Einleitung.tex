%!TEX root = ../Pflichtenheft.tex
\chapter{Einleitung}
\todo[inline]{Das kann man bestimmt ausschmücken... }
In modernen und inklusiven Arbeits- und Lernumgebungen spielen Rückzugsorte eine zentrale Rolle um produktives Arbeiten und das Wohlbefinden aller gewährleisten zu können.
Das Institut für Mensch-Maschine-Interaktion und Barrierefreiheit (MIB) am KIT bietet einen Ruheraum, welcher für Meetings, kurzen Pausen oder als Rückzugsort nach einer Reizüberflutung genutzt werden kann.

Derzeit wird der Status des Raumes durch ein einfaches Türschild angezeigt, welches lediglich den Status \textit{verfügbar} oder \textit{besetzt} anzeigt.
Dieses System bietet jedoch keine Möglichkeit, die Dringlichkeit sowie die Nutzung zu kommunizieren und zu planen.

Im Rahmen dieses Projekts wird ein innovatives Buchungssystem entwickelt, das über ein klassisches \textit{frei}- oder \textit{gebucht}-System hinausgeht.
Ziel ist es, ein System zu schaffen, das es den Nutzer*innen ermöglicht, ihre Bedürfnisse differenziert anzugeben.
Dadurch soll eine transparente und flexible Abstimmung über die Raumnutzung ermöglicht werden, die sowohl individuellen Bedürfnissen als auch einer optimalen Raumverteilung gerecht wird.