%!TEX root = ../Pflichtenheft.tex
\section{Allgemeine Hinweise}
\textbf{Template} Tragen Sie vor Beginn der Ausarbeitung Ihre Daten in die Datei common/teilnehmer.tex ein. Löschen Sie die Template-Texte und -Bilder, sobald Sie Ihren Text geschrieben haben.

\textbf{Anglizismen} Verwenden Sie ruhig die englischen Wörter.

\textbf{Fett-/Kursivschreibweise} Entscheiden Sie sich zu Beginn innerhalb der Gruppe, wann Sie \textit{kursive} und wann Sie \textbf{fette} Schrift einsetzen möchten.

\textbf{Bilder} Legen Sie eigene Bilder stets im figures Ordner ab. Nutzen Sie die im Template beispielhaft eingebundenen Bilder als Vorlage. Außerdem erlaubt Overleaf das Einfügen von Bildern im GUI (siehe obere Task-Leiste).


\chapter{Zielbestimmung}
\label{chap:target}
\improvement[inline]{Dieser Abschnitt sollte noch ausführlicher beschrieben werden.}

\section{Musskriterien}\label{sec:musskriterien}

\todo[inline]{Hier fallen uns noch sicherliche mehr Kriterien ein.}
\must{1}{Die Anwendung muss als Web-Applikation realisiert werden.}
\must{2}{Nutzer der Anwendung müssen in der Lage sein, sich mit ihrem KIT-Konto per \gls{OIDC} oder einem lokalen Gastkonto anzumelden.}
\must{3}{Es müssen die Ansichten Kalender, Termine und Termin-erstellen die Benutzerinteraktionen ermöglichen.}
\must{4}{Die Kalender-Ansicht muss einen klaren Überblick über bereits reservierte Zeiten geben.}
\must{5}{Die Kalender-Ansicht muss die Öffnungszeiten des Raumens zur Darstellung bringen.}
\must{6}{Die Kalender-Ansicht soll angemeldeten Nutzern ihre eigenen Termine hervorgehoben kennzeichnen.}
\must{7}{Die Termin-Ansicht muss die Möglichkeit bieten, genauere Informationen über einen Termin darzustellen.}
\must{8}{Die Termin-Erstellen-Ansicht muss die Möglichkeit bieten, einen Raum für eine gewisse Zeitperiode zu reservieren.}
\unsure{ist Kollaborativ ein guter Name dafür??}
\must{9}{Ereignisse müssen als kollaborativ gekennzeichnet werden können. Dies muss auf der Kalenderansicht ersichtlich sein.}

\unsure{what exactly should be specified??}
\must{10}{Bei der Reservierung eines Raumes muss die optionale Möglichkeit bestehen eine Beschreibung, Mail-Adresse (sollten NutzerInnen nicht mit ihrem KIT-Account angmeldet sein) zu hinterlegen. Hierbei muss der Nutzer klar darauf hingewiesen werden, wer diese Daten einsehen kann.}

\unsure{Specify more concretely, nuanced in what way?}
\must{11}{Die Priorität einer getätigten Reservierung muss nuanciert angegeben werden können.}
\unsure{Wie genau soll das passieren? Entwerfen wir hierfür einen Algo?}
\must{12}{Die Anwendung muss Terminkonflikte reibungslos mithilfe der Prioritäten lösen können.}
\unsure{Wie genau soll das passieren?}
\must{13}{Die Anwendung muss in der Lage sein, Nutzer darüber zu informieren, wenn ihre Buchung durch eine Buchung mit höherer Priorität überschrieben wurde.}
\must{14}{Nutzer der Anwendung müssen in der Lage sein, eine Reservierung zu stornieren.}
\unsure{Hier sollten wir genauer spezifizieren, wie wir die passworverwaltung umsetzen wollen.}
\must{15}{Es muss einen Admin-Nutzer geben, welcher sich per Passwort sich authentifiziert. Alleine der Serveradmin darf dieses Passwort ändern können.}
\must{16}{Der Admin muss Termine löschen, Gastnutzer deaktivieren, Räume bearbeiten und Öffnungszeiten einstellen können.}
\must{17}{Die Anwendung muss den Raumstatus möglichst unkompliziert darstellen können. Es muss also sowohl die aktuelle Belegegung als auch die Priorität (z.B. durch eine farbigen Banner angezeigt), zu sehen sein.}
\unsure[inline]{discuss the previous RM1 on Monday}

\section{Wunschkriterien}\label{sec:wunschkriterien}

\todo[inline]{Hier fallen uns noch sicherlich mehr Kriterien ein.}

\wish{1}{Es könnte die Möglichkeit geben mehr als einen Raum zur Buchung anzubieten. Dabei kann eine Arbeitsraum-Auswahl vor der Kalender Ansicht dem Nutzer die verschiedenen Möglichkeiten Präsentieren. Existiert nur ein Arbeitsraum wird diese Auswahl Übersprungen.}
\wish{2}{In der Kalenderansicht könnten Feiertage automatisch eingebunden werden.}
\wish{3}{Admins könnten die Möglichkeit haben geplante Wartungs- und Sperrzeiten einzurichten.}
\wish{4}{Termine könnten nach dem Einrichten per \gls{iCal}-Format exportiert werden können.}
\wish{5}{Nutzer könnten bei der Ereignis-Ansicht die Möglichkeit haben, ihre eigenen Termine zu editieren.}
\wish{6}{Die Kalenderansicht könnte visualisieren, welche Termine bereits in der Vergangenheit liegen und wo der Übergang von der Vergangenheit zur Zukunft liegt.}
\wish{7}{Tooltips könnten Nutzern erklären wofür bestimmte Elemente der \gls{UI} sind.}
\wish{9}{Es könnte in Hardware einen Panik-Button und Hinweise zum aktuellen Raumstatus geben.}
\wish{10}{Die Anwendung könnte in der Lage sein, Nutzer zu informieren, sollte der gewünschte Termin frei werden.}
\must{11}{Es könnte einen Quick-checkout-button geben.}


\section{Abgrenzungskriterien}\label{sec:abgrenzungskriterien}

\wont{1}{Die Verteilung von Buchungen zwischen ähnlichen Räumen ist nicht Teil des Projekts.}
\wont{2}{Das Skalieren der Anwendung auf große Raumzahlen ist nicht vorgesehen.}
\wont{3}{Die Buchung von Räumen für mehrere Tage ist nicht vorgesehen.}
\wont{4}{Die Verwaltung von Räumen, die mehrere Arbeitsplätze umfassen, ist nicht vorgesehen.}
\wont{5}{Die Entwicklung platfformspezifischer Anwendungen und der dafür notwendigen \gls{API}s ist nicht vorgesehen.}
\wont{6}{Die Reservierung ist nur für die nahe Zukunft gedacht, also keine Langzeitplanung.}