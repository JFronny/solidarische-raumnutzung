%!TEX root = ../Pflichtenheft.tex

\chapter{Zielbestimmung}
\label{chap:target}


\section{Musskriterien}\label{sec:musskriterien}

\must{1}{Die Anwendung muss als Web-Applikation realisiert werden.}
\must{2}{Nutzende der Anwendung müssen sich mit ihrem KIT-Konto per \gls{OIDC} oder einem lokalen Gastkonto anmelden können.}
\must{3}{Nutzende müssen sich abmelden können.}
\must{4}{Die Anwendung muss die Ansichten Kalender, Termin, Termin-erstellen, Login, Kontenliste und Terminübersicht anbieten.}
\must{5}{Die Ansicht <Kalender> muss einen klaren Überblick über die bereits reservierten Zeiten geben.}
\must{6}{Die Ansicht <Kalender> muss die Öffnungszeiten des Raumes darstellen.}
\must{7}{Die Ansicht <Kalender> muss die Termine des angemeldeten Nutzenden hervorgehoben darstellen.}
\must{8}{Die Ansicht <Termin> muss die Möglichkeit bieten, genauere Informationen über einen Termin darzustellen.}
\must{9}{Die Ansicht <Termin-Erstellen> muss die Möglichkeit bieten, einen Raum für eine bestimmte Zeitperiode zu reservieren.}
\must{10}{Bei der Reservierung eines Raumes muss optional die Möglichkeit bestehen eine Beschreibung zu hinterlegen. Hierbei müssen die Nutzenden klar darauf hingewiesen werden, wer diese Daten einsehen kann.}
\must{11}{Die Terminübersicht muss den Nutzenden die Möglichkeit bieten, all ihre Termine zu verwalten.}
\must{12}{Die Priorität eines Termins ist in drei Stufen gegliedert.}
\must{13}{Beim Erstellen eines Termins wählen die Nutzenden aus, ob sie ihren Raum mit anderen Personen teilen möchten. Dabei stehen die Optionen \textit{Ja}, \textit{Nein} und \textit{Auf Anfrage} zur Auswahl.}
\must{14}{Die Anwendung muss Terminkonflikte reibungslos mithilfe der Prioritäten lösen können. Dabei überschreiben Termine mit höherer Priorität andere.}
\must{15}{Die Anwendung muss in der Lage sein, Nutzende per E-Mail darüber zu informieren, wenn ihr Termin durch einen Termin mit höherer Priorität überschrieben wurde.}
\must{16}{Nutzende der Anwendung müssen in der Lage sein, eine Reservierung zu stornieren.}
\must{17}{Es muss einen Admin-Nutzer geben, welcher sich per Passwort authentifiziert. Nur der Server-Admin darf dieses Passwort ändern können.}
\must{18}{Die Ansicht <Kontoliste> muss den Admins die Möglichkeit bieten, einzelne Konten sowie die Anmeldung per Gastkonto zu deaktivieren.}
\must{19}{Admins müssen in der Ansicht <Termin> Termine löschen können.}
\must{20}{Es muss ein farbcodiertes Banner geben, der den aktuellen Status des Raumes anzeigt.}
\must{21}{Admins müssen in der Ansicht <Kalender> die Öffnungszeiten einstellen können}


\section{Wunschkriterien}\label{sec:wunschkriterien}

\wish{1}{Es könnte die Möglichkeit geben, mehr als einen Raum zur Buchung anzubieten. Dabei könnte eine Raumauswahl vor der Ansicht <Kalender> die verschiedenen Möglichkeiten präsentieren. Existiert nur ein Raum, wird diese Auswahl übersprungen.}
\wish{2}{In der Ansicht <Kalender> könnten Feiertage automatisch eingebunden werden.}
\wish{3}{Admins könnten die Möglichkeit haben, geplante Wartungs- und Sperrzeiten einzurichten.}
\wish{4}{Termine könnten nach der Buchung im \gls{iCal}-Format zum Export angeboten werden.}
\wish{5}{Nutzende könnten in der Ansicht <Termin> die Möglichkeit haben, ihre eigenen Termine zu bearbeiten.}
\wish{6}{Die Ansicht <Kalender> könnte visualisieren, welche Termine bereits in der Vergangenheit liegen und wo der Übergang von der Vergangenheit zur Zukunft liegt.}
\wish{7}{Tooltips könnten Nutzenden erklären, wofür bestimmte Elemente der \gls{UI} verwendet werden.}
\wish{9}{Es könnte einen physischen Panik-Button geben.}
\wish{10}{Die Anwendung könnte in der Lage sein, Nutzende zu informieren, falls ein gewünschter Termin frei wird.}
\wish{11}{Es könnte einen Quick-Checkin-Button geben, welcher eine vorausgefüllte Terminerstellung öffnet. Dieser bietet auch eine Alternative zur Interaktion mit dem Kalender.}
\wish{12}{Es könnte einen Quick-Checkout-Button geben, der vorzeitiges Beenden eines Termines ermöglicht.}
\wish{13}{Admins könnten eine Statistik-Ansicht nutzen.}


\section{Abgrenzungskriterien}\label{sec:abgrenzungskriterien}

\wont{1}{Die Verteilung von Buchungen zwischen ähnlichen Räumen ist nicht Teil des Projekts.}
\wont{2}{Das Skalieren der Anwendung auf eine große Anzahl von Räumen ist nicht vorgesehen.}
\wont{3}{Die Buchung von Räumen für mehrere Tage ist nicht vorgesehen.}
\wont{4}{Die Verwaltung von Räumen, die mehrere Arbeitsplätze umfassen, ist nicht vorgesehen.}
\wont{5}{Die Entwicklung plattformspezifischer Anwendungen und der dafür notwendigen \gls{API}s ist nicht vorgesehen.}
\wont{6}{Die Reservierung ist nur für die nahe Zukunft gedacht, eine Langzeitplanung ist nicht vorgesehen.}
\wont{7}{E-Mail-Adressen von Gastkonten werden nicht verifiziert.}