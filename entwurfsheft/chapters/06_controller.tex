%!TEX root = ../main.tex

\chapter{Controller}
\label{ch:controller}

Das Controller-Layer der Anwendung ist für die Verarbeitung von Anfragen und die Erzeugung von Antworten zuständig.
Es dekodiert und prüft dabei in \gls{HTTP}-Anfragen enthaltene Eingaben und generiert Parameter für die \gls{JTE}-Templates der View.
Zur Bearbeitung der überprüften Anfragen werden die Services des Service-Layers genutzt, welche die Datenverarbeitung kapseln.
Da einige Endpunkte eine Anmeldung benötigen, werden unangemeldete Nutzende, welche versuchen, mit einem solchen Endpunkt zu interagieren, automatisch auf die Anmeldeseite weitergeleitet.
Das Controller-Layer bietet zur Interaktion durch Nutzer die folgenden Endpunkte, welche die \gls{HTTP}-Oberfläche der Anwendung kapseln und die im Pflichtenheft beschriebenen Ansichten modellieren.

Für die Ansicht \textit{Kalender} (implementiert in \hyperref[edu.kit.hci.soli.controller.CalendarController]{\texttt{CalendarController}}) werden folgende Endpunkte bereitgestellt:
\begin{itemize}
    \item \texttt{GET /} gibt alle Buchungen für den Standardraum mit der fixierten ID \texttt{1} zurück.
    \item \texttt{GET /\{roomId\}} gibt die Kalenderansicht für den Raum mit der ID \texttt{roomId} zurück.
          Als Implementation für den Kalender wird dabei \gls{FullCalendar} genutzt.
    \item \texttt{GET /api/\{roomId\}/events} gibt alle Buchungen für einen Raum im \gls{FullCalendar}-Format zurück.
          Dieser Endpunkt wird von \gls{FullCalendar} genutzt, um die Buchungen im Kalender anzuzeigen.
          Links zur Ansicht \texttt{Termin} werden für die Termine eingebettet.
          Da es sich bei diesem Endpunkt um einen \gls{REST}-Endpunkt handelt, ist er separat in \hyperref[edu.kit.hci.soli.controller.EventFeedController]{\texttt{EventFeedController}} implementiert.
\end{itemize}

Für die Ansichten \textit{Termin} und \textit{Terminübersicht} (implementiert in \hyperref[edu.kit.hci.soli.controller.BookingViewController]{\texttt{BookingViewController}}) werden folgende Endpunkte bereitgestellt:
\begin{itemize}
    \item \texttt{GET /\{roomId\}/bookings} gibt die Terminübersicht des angemeldeten Nutzers für den Raum mit der ID \texttt{roomId} zurück.
    \item \texttt{GET /\{roomId\}/bookings/\{eventId\}} gibt die Ansicht \textit{Termin} für den Termin \texttt{eventId} zurück.
          Falls ein Administrator oder die Person, welche den Termin erstellte, angemeldet ist, wird eine Option, den Termin zu löschen, als Link eingebettet.
          Falls der Termin als kollaborativ markiert ist, wird ein link zur Ansicht \textit{Termin-Erstellen} mit den Start- und Endzeiten des angewählten Termins eingebettet.
    \item \texttt{DELETE /\{roomId\}/bookings/\{eventId\}} löscht den Termin \texttt{eventId} und leitet Nutzende auf die Terminübersicht weiter.
\end{itemize}

Für die Ansicht \textit{Termin-Erstellen} (implementiert in \hyperref[edu.kit.hci.soli.controller.BookingCreateController]{\texttt{BookingCreateController}}) werden folgende Endpunkte bereitgestellt:
\begin{itemize}
    \item \texttt{GET /\{roomId\}/bookings/new} gibt das Formular (implementiert als \gls{HTML-Form}) zur Erstellung eines neuen Termins für den Raum mit der ID \texttt{roomId} zurück.
    \item \texttt{POST /\{roomId\}/bookings/new} erstellt einen neuen Termin für den Raum mit der ID \texttt{roomId} und leitet Nutzende auf die Ansicht \textit{Termin} weiter.
          Die Form-Eingaben des Formulars werden dabei dekodiert und geprüft.
          Falls ein Konflikt mit einem bereits bestehenden Termin besteht, werden Nutzende auf die Konfliktseite weitergeleitet.
          In diesem Fall wird der dekodierte Termin als \hyperref[edu.kit.hci.soli.domain.Booking]{\texttt{Booking}} in der Session gespeichert.
    \item \texttt{POST /\{roomId\}/bookings/new/conflict} bietet Nutzenden die Möglichkeit, einen Konflikt entsprechend der Spezifikation im Pflichtenheft zu lösen.
          Dabei wird eine mögliche Lösung vorgeschlagen, welche Nutzende bestätigen oder ablehnen können.
          Im Falle der Bestätigung wird der in der Session gespeicherte Termin erstellt und Nutzende auf die Ansicht \textit{Termin} weitergeleitet.
          War unter den Konflikten ein Termin, welcher Zustimmung zur Kollaboration benötigt, wird entsprechend eine E-Mail an die betroffenen Nutzenden versendet.
          War unter den Konflikten ein Termin, welcher durch die Erstellung gelöscht wurde, wird entsprechend eine E-Mail an die betroffenen Nutzenden versendet.
\end{itemize}

Für die Ansicht \textit{Login} (implementiert in \hyperref[edu.kit.hci.soli.controller.LoginController]{\texttt{LoginController}}) werden folgende Endpunkte bereitgestellt:
\begin{itemize}
    \item \texttt{GET /login} gibt das Formular zur Anmeldung zurück. Die Anmeldung kann mit einem eingebetteten Formular als Administrator oder mit einem Link als Gast oder über \gls{OIDC} erfolgen.
          Für die Anmeldung als Administrator wird eine \gls{HTML-Form} genutzt, welche die Eingabe eines Passworts fordert.
    \item \texttt{POST /login} erlaubt die Anmeldung mit lokalen Anmeldedaten (also als Gast oder Administrator) und leitet Nutzende auf die Ansicht, welche die Anmeldung verursachte, oder die Ansicht \textit{Kalender} weiter.
          Dieser Endpunkt wird bei Bestätigung einer Gast- oder Administratoren-Anmeldung aufgerufen.
    \item \texttt{GET /login/guest} gibt das Formular zur Anmeldung als Gast zurück.
    \item \texttt{GET /oauth2/authorization/kit} verarbeitet die Anmeldung über \gls{OIDC} und leitet Nutzende auf die Ansicht, welche die Anmeldung verursachte, oder die Ansicht \textit{Kalender} weiter.
          Auf diesen Endpunkt werden Nutzende bei Bestätigung einer \gls{OIDC}-Anmeldung durch den OIDC-Server weitergeleitet.
\end{itemize}

Für die Ansicht \textit{Kontoliste} (implementiert in \hyperref[edu.kit.hci.soli.controller.UsersController]{\texttt{UsersController}}) werden folgende Endpunkte bereitgestellt:
\begin{itemize}
    \item \texttt{GET /admin/users} gibt die Kontoliste aller Nutzenden zurück. Das Administrationskonto wird dabei nicht angezeigt.
          Für jedes Konto wird ein Link zur Deaktivierung als \gls{HTML-Form} eingebettet (dies erlaubt die Nutzung von \texttt{PUT}-Anfragen).
    \item \texttt{PUT /admin/users/{userId}/deactivate} deaktiviert das Konto mit der ID \texttt{userId} und leitet Nutzende auf die Kontoliste weiter.
          Falls ein Konto bereits deaktiviert ist, wird es reaktiviert.
\end{itemize}

Außerdem werden folgende allgemeine Endpunkte (implementiert in \hyperref[edu.kit.hci.soli.controller.MainController]{\texttt{MainController}}) bereitgestellt:
\begin{itemize}
    \item \texttt{GET /error} gibt eine Fehlerseite zurück. Dieser Endpunkt wird von Spring automatisch aufgerufen, wenn ein Fehler auftritt.
    \item \texttt{GET /disabled} gibt eine Seite zurück, die Nutzende darüber informiert, dass ihr Konto deaktiviert wurde.
\end{itemize}



\section{Package edu.kit.hci.soli.controller}
\begin{adjustwidth}{1.5em}{0pt}
  \subsection{Class BookingCreateController\label{edu.kit.hci.soli.controller.BookingCreateController} }
  \begin{adjustwidth}{1.5em}{0pt}
    {\begin{tabularx}{1.0\linewidth}{@{}l R@{}}
      \emph{Signature} & \texttt{public  class \texttt{\hyperref[edu.kit.hci.soli.controller.BookingCreateController]{\texttt{BookingCreateController}}}} \\
      \hline
      \emph{Behaviour} & Controller for handling booking creation requests.  \\
      \hline
  
    \end{tabularx}}\subsubsection{Constructor\label{edu.kit.hci.soli.controller.BookingCreateController@edu.kit.hci.soli.controller.BookingCreateController(edu.kit.hci.soli.service.BookingsService,edu.kit.hci.soli.service.RoomService)}}
    \begin{adjustwidth}{1.5em}{0pt}
      {\begin{tabularx}{1.0\linewidth}{@{}l R@{}}
        \emph{Signature} & \texttt{public \texttt{\hyperref[edu.kit.hci.soli.controller.BookingCreateController]{\texttt{BookingCreateController}}}(\texttt{\hyperref[edu.kit.hci.soli.service.BookingsService]{\texttt{BookingsService}}} bookingsService, \texttt{\hyperref[edu.kit.hci.soli.service.RoomService]{\texttt{RoomService}}} roomService)} \\
        \hline
        \emph{Behaviour} & Constructs a BookingCreateController with the specified services.    \\
        \hline
        \emph{Parameters} & {\begin{tabularx}{1.0\linewidth}{@{}l R@{}}
          \texttt{bookingsService}: & the service for managing bookings  \\
          \texttt{roomService}: &     the service for managing rooms  \\
  
        \end{tabularx}} \\
        \hline
  
      \end{tabularx}}
    \end{adjustwidth}\subsubsection{Method resolveConflict\label{edu.kit.hci.soli.controller.BookingCreateController@resolveConflict(org.springframework.ui.Model,jakarta.servlet.http.HttpServletRequest,java.lang.Long)}}
    \begin{adjustwidth}{1.5em}{0pt}
      {\begin{tabularx}{1.0\linewidth}{@{}l R@{}}
        \emph{Signature} & \texttt{public \texttt{String} resolveConflict(\texttt{Model} model, \texttt{HttpServletRequest} request, \texttt{Long} roomId)} \\
        \hline
        \emph{Behaviour} & Resolves a booking conflict.    \\
        \hline
        \emph{Returns} & the view name  \\
        \hline
        \emph{Parameters} & {\begin{tabularx}{1.0\linewidth}{@{}l R@{}}
          \texttt{model}: &   the model to be used in the view  \\
          \texttt{request}: & the HTTP request  \\
          \texttt{roomId}: &  the ID of the room  \\
  
        \end{tabularx}} \\
        \hline
  
      \end{tabularx}}
    \end{adjustwidth}\subsubsection{Method createBooking\label{edu.kit.hci.soli.controller.BookingCreateController@createBooking(org.springframework.ui.Model,jakarta.servlet.http.HttpServletResponse,jakarta.servlet.http.HttpServletRequest,java.lang.Long,edu.kit.hci.soli.config.security.SoliUserDetails,edu.kit.hci.soli.controller.BookingCreateController.FormData)}}
    \begin{adjustwidth}{1.5em}{0pt}
      {\begin{tabularx}{1.0\linewidth}{@{}l R@{}}
        \emph{Signature} & \texttt{public \texttt{String} createBooking(\texttt{Model} model, \texttt{HttpServletResponse} response, \texttt{HttpServletRequest} request, \texttt{Long} roomId, \texttt{\hyperref[edu.kit.hci.soli.config.security.SoliUserDetails]{\texttt{SoliUserDetails}}} principal, \texttt{\hyperref[edu.kit.hci.soli.controller.BookingCreateController.FormData]{\texttt{FormData}}} formData)} \\
        \hline
        \emph{Behaviour} & Creates a new booking.    \\
        \hline
        \emph{Returns} & the view name  \\
        \hline
        \emph{Parameters} & {\begin{tabularx}{1.0\linewidth}{@{}l R@{}}
          \texttt{model}: &     the model to be used in the view  \\
          \texttt{response}: &  the HTTP response  \\
          \texttt{request}: &   the HTTP request  \\
          \texttt{roomId}: &    the ID of the room  \\
          \texttt{principal}: & the authenticated user details  \\
          \texttt{formData}: &  the form data for the booking  \\
  
        \end{tabularx}} \\
        \hline
  
      \end{tabularx}}
    \end{adjustwidth}\subsubsection{Method newBooking\label{edu.kit.hci.soli.controller.BookingCreateController@newBooking(org.springframework.ui.Model,jakarta.servlet.http.HttpServletResponse,java.lang.Long,java.time.LocalDateTime,java.time.LocalDateTime,java.lang.Boolean)}}
    \begin{adjustwidth}{1.5em}{0pt}
      {\begin{tabularx}{1.0\linewidth}{@{}l R@{}}
        \emph{Signature} & \texttt{public \texttt{String} newBooking(\texttt{Model} model, \texttt{HttpServletResponse} response, \texttt{Long} roomId, \texttt{LocalDateTime} start, \texttt{LocalDateTime} end, \texttt{Boolean} cooperative)} \\
        \hline
        \emph{Behaviour} & Displays the form for creating a new booking.    \\
        \hline
        \emph{Returns} & the view name  \\
        \hline
        \emph{Parameters} & {\begin{tabularx}{1.0\linewidth}{@{}l R@{}}
          \texttt{model}: &       the model to be used in the view  \\
          \texttt{response}: &    the HTTP response  \\
          \texttt{roomId}: &      the ID of the room  \\
          \texttt{start}: &       the start date and time of the booking  \\
          \texttt{end}: &         the end date and time of the booking  \\
          \texttt{cooperative}: & whether the booking is cooperative  \\
  
        \end{tabularx}} \\
        \hline
  
      \end{tabularx}}
    \end{adjustwidth}\subsubsection{Class FormData\label{edu.kit.hci.soli.controller.BookingCreateController.FormData} }
    \begin{adjustwidth}{1.5em}{0pt}
      {\begin{tabularx}{1.0\linewidth}{@{}l R@{}}
        \emph{Signature} & \texttt{public static  class \texttt{\hyperref[edu.kit.hci.soli.controller.BookingCreateController.FormData]{\texttt{FormData}}}} \\
        \hline
        \emph{Behaviour} & Data class for event creation form data.  \\
        \hline
  
      \end{tabularx}}\paragraph{Constructor\label{edu.kit.hci.soli.controller.BookingCreateController.FormData@edu.kit.hci.soli.controller.BookingCreateController.FormData()}}
      \begin{adjustwidth}{1.5em}{0pt}
        {\begin{tabularx}{1.0\linewidth}{@{}l R@{}}
          \emph{Signature} & \texttt{public \texttt{\hyperref[edu.kit.hci.soli.controller.BookingCreateController.FormData]{\texttt{FormData}}}()} \\
          \hline
  
        \end{tabularx}}
      \end{adjustwidth}\paragraph{Field cooperative\label{edu.kit.hci.soli.controller.BookingCreateController.FormData@cooperative}}
      \begin{adjustwidth}{1.5em}{0pt}
        {\begin{tabularx}{1.0\linewidth}{@{}l R@{}}
          \emph{Signature} & \texttt{public \texttt{\hyperref[edu.kit.hci.soli.controller.BookingCreateController.FormData]{\texttt{FormData}}} cooperative} \\
          \hline
  
        \end{tabularx}}
      \end{adjustwidth}\paragraph{Field priority\label{edu.kit.hci.soli.controller.BookingCreateController.FormData@priority}}
      \begin{adjustwidth}{1.5em}{0pt}
        {\begin{tabularx}{1.0\linewidth}{@{}l R@{}}
          \emph{Signature} & \texttt{public \texttt{\hyperref[edu.kit.hci.soli.controller.BookingCreateController.FormData]{\texttt{FormData}}} priority} \\
          \hline
  
        \end{tabularx}}
      \end{adjustwidth}\paragraph{Field description\label{edu.kit.hci.soli.controller.BookingCreateController.FormData@description}}
      \begin{adjustwidth}{1.5em}{0pt}
        {\begin{tabularx}{1.0\linewidth}{@{}l R@{}}
          \emph{Signature} & \texttt{public \texttt{\hyperref[edu.kit.hci.soli.controller.BookingCreateController.FormData]{\texttt{FormData}}} description} \\
          \hline
  
        \end{tabularx}}
      \end{adjustwidth}\paragraph{Field end\label{edu.kit.hci.soli.controller.BookingCreateController.FormData@end}}
      \begin{adjustwidth}{1.5em}{0pt}
        {\begin{tabularx}{1.0\linewidth}{@{}l R@{}}
          \emph{Signature} & \texttt{public \texttt{\hyperref[edu.kit.hci.soli.controller.BookingCreateController.FormData]{\texttt{FormData}}} end} \\
          \hline
  
        \end{tabularx}}
      \end{adjustwidth}\paragraph{Field start\label{edu.kit.hci.soli.controller.BookingCreateController.FormData@start}}
      \begin{adjustwidth}{1.5em}{0pt}
        {\begin{tabularx}{1.0\linewidth}{@{}l R@{}}
          \emph{Signature} & \texttt{public \texttt{\hyperref[edu.kit.hci.soli.controller.BookingCreateController.FormData]{\texttt{FormData}}} start} \\
          \hline
  
        \end{tabularx}}
      \end{adjustwidth}
    \end{adjustwidth}
  \end{adjustwidth}\subsection{Class BookingViewController\label{edu.kit.hci.soli.controller.BookingViewController} }
  \begin{adjustwidth}{1.5em}{0pt}
    {\begin{tabularx}{1.0\linewidth}{@{}l R@{}}
      \emph{Signature} & \texttt{public  class \texttt{\hyperref[edu.kit.hci.soli.controller.BookingViewController]{\texttt{BookingViewController}}}} \\
      \hline
      \emph{Behaviour} & Controller for handling booking-related requests.  \\
      \hline
  
    \end{tabularx}}\subsubsection{Constructor\label{edu.kit.hci.soli.controller.BookingViewController@edu.kit.hci.soli.controller.BookingViewController(edu.kit.hci.soli.service.BookingsService,edu.kit.hci.soli.service.RoomService,edu.kit.hci.soli.service.UserService)}}
    \begin{adjustwidth}{1.5em}{0pt}
      {\begin{tabularx}{1.0\linewidth}{@{}l R@{}}
        \emph{Signature} & \texttt{public \texttt{\hyperref[edu.kit.hci.soli.controller.BookingViewController]{\texttt{BookingViewController}}}(\texttt{\hyperref[edu.kit.hci.soli.service.BookingsService]{\texttt{BookingsService}}} bookingsService, \texttt{\hyperref[edu.kit.hci.soli.service.RoomService]{\texttt{RoomService}}} roomService, \texttt{\hyperref[edu.kit.hci.soli.service.UserService]{\texttt{UserService}}} userService)} \\
        \hline
        \emph{Behaviour} & Constructs a BookingViewController with the specified services.    \\
        \hline
        \emph{Parameters} & {\begin{tabularx}{1.0\linewidth}{@{}l R@{}}
          \texttt{bookingsService}: & the service for managing bookings  \\
          \texttt{roomService}: &     the service for managing rooms  \\
          \texttt{userService}: &     the service for managing users  \\
  
        \end{tabularx}} \\
        \hline
  
      \end{tabularx}}
    \end{adjustwidth}\subsubsection{Method viewEvent\label{edu.kit.hci.soli.controller.BookingViewController@viewEvent(org.springframework.ui.Model,jakarta.servlet.http.HttpServletResponse,edu.kit.hci.soli.config.security.SoliUserDetails,java.lang.Long,java.lang.Long)}}
    \begin{adjustwidth}{1.5em}{0pt}
      {\begin{tabularx}{1.0\linewidth}{@{}l R@{}}
        \emph{Signature} & \texttt{public \texttt{String} viewEvent(\texttt{Model} model, \texttt{HttpServletResponse} response, \texttt{\hyperref[edu.kit.hci.soli.config.security.SoliUserDetails]{\texttt{SoliUserDetails}}} principal, \texttt{Long} roomId, \texttt{Long} eventId)} \\
        \hline
        \emph{Behaviour} & Displays the details of a specific event.    \\
        \hline
        \emph{Returns} & the view name  \\
        \hline
        \emph{Parameters} & {\begin{tabularx}{1.0\linewidth}{@{}l R@{}}
          \texttt{model}: &     the model to be used in the view  \\
          \texttt{response}: &  the HTTP response  \\
          \texttt{principal}: & the authenticated user details  \\
          \texttt{roomId}: &    the ID of the room  \\
          \texttt{eventId}: &   the ID of the event  \\
  
        \end{tabularx}} \\
        \hline
  
      \end{tabularx}}
    \end{adjustwidth}\subsubsection{Method roomBookings\label{edu.kit.hci.soli.controller.BookingViewController@roomBookings(org.springframework.ui.Model,jakarta.servlet.http.HttpServletResponse,edu.kit.hci.soli.config.security.SoliUserDetails,java.lang.Long)}}
    \begin{adjustwidth}{1.5em}{0pt}
      {\begin{tabularx}{1.0\linewidth}{@{}l R@{}}
        \emph{Signature} & \texttt{public \texttt{String} roomBookings(\texttt{Model} model, \texttt{HttpServletResponse} response, \texttt{\hyperref[edu.kit.hci.soli.config.security.SoliUserDetails]{\texttt{SoliUserDetails}}} principal, \texttt{Long} roomId)} \\
        \hline
        \emph{Behaviour} & Displays the bookings for a specific room.    \\
        \hline
        \emph{Returns} & the view name  \\
        \hline
        \emph{Parameters} & {\begin{tabularx}{1.0\linewidth}{@{}l R@{}}
          \texttt{model}: &     the model to be used in the view  \\
          \texttt{response}: &  the HTTP response  \\
          \texttt{principal}: & the authenticated user details  \\
          \texttt{roomId}: &    the ID of the room  \\
  
        \end{tabularx}} \\
        \hline
  
      \end{tabularx}}
    \end{adjustwidth}\subsubsection{Method deleteBookings\label{edu.kit.hci.soli.controller.BookingViewController@deleteBookings(org.springframework.ui.Model,jakarta.servlet.http.HttpServletResponse,edu.kit.hci.soli.config.security.SoliUserDetails,java.lang.Long,java.lang.Long)}}
    \begin{adjustwidth}{1.5em}{0pt}
      {\begin{tabularx}{1.0\linewidth}{@{}l R@{}}
        \emph{Signature} & \texttt{public \texttt{String} deleteBookings(\texttt{Model} model, \texttt{HttpServletResponse} response, \texttt{\hyperref[edu.kit.hci.soli.config.security.SoliUserDetails]{\texttt{SoliUserDetails}}} principal, \texttt{Long} roomId, \texttt{Long} eventId)} \\
        \hline
        \emph{Behaviour} & Deletes a booking for a specific room and event.    \\
        \hline
        \emph{Returns} & the view name  \\
        \hline
        \emph{Parameters} & {\begin{tabularx}{1.0\linewidth}{@{}l R@{}}
          \texttt{model}: &     the model to be used in the view  \\
          \texttt{response}: &  the HTTP response  \\
          \texttt{principal}: & the authenticated user details  \\
          \texttt{roomId}: &    the ID of the room  \\
          \texttt{eventId}: &   the ID of the event  \\
  
        \end{tabularx}} \\
        \hline
  
      \end{tabularx}}
    \end{adjustwidth}
  \end{adjustwidth}\subsection{Class CalendarController\label{edu.kit.hci.soli.controller.CalendarController} }
  \begin{adjustwidth}{1.5em}{0pt}
    {\begin{tabularx}{1.0\linewidth}{@{}l R@{}}
      \emph{Signature} & \texttt{public  class \texttt{\hyperref[edu.kit.hci.soli.controller.CalendarController]{\texttt{CalendarController}}}} \\
      \hline
      \emph{Behaviour} & Controller for handling calendar-related requests.  \\
      \hline
  
    \end{tabularx}}\subsubsection{Constructor\label{edu.kit.hci.soli.controller.CalendarController@edu.kit.hci.soli.controller.CalendarController(edu.kit.hci.soli.service.RoomService)}}
    \begin{adjustwidth}{1.5em}{0pt}
      {\begin{tabularx}{1.0\linewidth}{@{}l R@{}}
        \emph{Signature} & \texttt{public \texttt{\hyperref[edu.kit.hci.soli.controller.CalendarController]{\texttt{CalendarController}}}(\texttt{\hyperref[edu.kit.hci.soli.service.RoomService]{\texttt{RoomService}}} roomService)} \\
        \hline
        \emph{Behaviour} & Constructs a CalendarController with the specified  \texttt{\hyperref[edu.kit.hci.soli.service.RoomService]{\texttt{RoomService}}}.    \\
        \hline
        \emph{Parameters} & {\begin{tabularx}{1.0\linewidth}{@{}l R@{}}
          \texttt{roomService}: & the service for managing rooms  \\
  
        \end{tabularx}} \\
        \hline
  
      \end{tabularx}}
    \end{adjustwidth}\subsubsection{Method calendar\label{edu.kit.hci.soli.controller.CalendarController@calendar(org.springframework.ui.Model,jakarta.servlet.http.HttpServletRequest,jakarta.servlet.http.HttpServletResponse,edu.kit.hci.soli.config.security.SoliUserDetails,long)}}
    \begin{adjustwidth}{1.5em}{0pt}
      {\begin{tabularx}{1.0\linewidth}{@{}l R@{}}
        \emph{Signature} & \texttt{public \texttt{String} calendar(\texttt{Model} model, \texttt{HttpServletRequest} request, \texttt{HttpServletResponse} response, \texttt{\hyperref[edu.kit.hci.soli.config.security.SoliUserDetails]{\texttt{SoliUserDetails}}} principal, \texttt{long} roomId)} \\
        \hline
        \emph{Behaviour} & Displays the calendar for a specific room.    \\
        \hline
        \emph{Returns} & the view name  \\
        \hline
        \emph{Parameters} & {\begin{tabularx}{1.0\linewidth}{@{}l R@{}}
          \texttt{model}: &     the model to be used in the view  \\
          \texttt{request}: &   the HTTP request  \\
          \texttt{response}: &  the HTTP response  \\
          \texttt{principal}: & the authenticated user details  \\
          \texttt{roomId}: &    the ID of the room  \\
  
        \end{tabularx}} \\
        \hline
  
      \end{tabularx}}
    \end{adjustwidth}\subsubsection{Method calendar\label{edu.kit.hci.soli.controller.CalendarController@calendar(org.springframework.ui.Model,jakarta.servlet.http.HttpServletRequest,jakarta.servlet.http.HttpServletResponse,edu.kit.hci.soli.config.security.SoliUserDetails)}}
    \begin{adjustwidth}{1.5em}{0pt}
      {\begin{tabularx}{1.0\linewidth}{@{}l R@{}}
        \emph{Signature} & \texttt{public \texttt{String} calendar(\texttt{Model} model, \texttt{HttpServletRequest} request, \texttt{HttpServletResponse} response, \texttt{\hyperref[edu.kit.hci.soli.config.security.SoliUserDetails]{\texttt{SoliUserDetails}}} principal)} \\
        \hline
        \emph{Behaviour} & Displays the calendar for the default room.    \\
        \hline
        \emph{Returns} & the view name  \\
        \hline
        \emph{Parameters} & {\begin{tabularx}{1.0\linewidth}{@{}l R@{}}
          \texttt{model}: &     the model to be used in the view  \\
          \texttt{request}: &   the HTTP request  \\
          \texttt{response}: &  the HTTP response  \\
          \texttt{principal}: & the authenticated user details  \\
  
        \end{tabularx}} \\
        \hline
  
      \end{tabularx}}
    \end{adjustwidth}
  \end{adjustwidth}\subsection{Class EventFeedController\label{edu.kit.hci.soli.controller.EventFeedController} }
  \begin{adjustwidth}{1.5em}{0pt}
    {\begin{tabularx}{1.0\linewidth}{@{}l R@{}}
      \emph{Signature} & \texttt{public  class \texttt{\hyperref[edu.kit.hci.soli.controller.EventFeedController]{\texttt{EventFeedController}}}} \\
      \hline
      \emph{Behaviour} & REST controller for generating the FullCalendar event feed.  \\
      \hline
  
    \end{tabularx}}\subsubsection{Constructor\label{edu.kit.hci.soli.controller.EventFeedController@edu.kit.hci.soli.controller.EventFeedController(edu.kit.hci.soli.service.BookingsService,edu.kit.hci.soli.service.RoomService)}}
    \begin{adjustwidth}{1.5em}{0pt}
      {\begin{tabularx}{1.0\linewidth}{@{}l R@{}}
        \emph{Signature} & \texttt{public \texttt{\hyperref[edu.kit.hci.soli.controller.EventFeedController]{\texttt{EventFeedController}}}(\texttt{\hyperref[edu.kit.hci.soli.service.BookingsService]{\texttt{BookingsService}}} bookingsRepository, \texttt{\hyperref[edu.kit.hci.soli.service.RoomService]{\texttt{RoomService}}} roomService)} \\
        \hline
        \emph{Behaviour} & Constructs an EventFeedController with the specified  \texttt{\hyperref[edu.kit.hci.soli.service.BookingsService]{\texttt{BookingsService}}}.    \\
        \hline
        \emph{Parameters} & {\begin{tabularx}{1.0\linewidth}{@{}l R@{}}
          \texttt{bookingsRepository}: & the service for managing bookings  \\
          \texttt{roomService}: &        the service for managing rooms  \\
  
        \end{tabularx}} \\
        \hline
  
      \end{tabularx}}
    \end{adjustwidth}\subsubsection{Method handleIllegalArgumentException\label{edu.kit.hci.soli.controller.EventFeedController@handleIllegalArgumentException(java.lang.IllegalArgumentException)}}
    \begin{adjustwidth}{1.5em}{0pt}
      {\begin{tabularx}{1.0\linewidth}{@{}l R@{}}
        \emph{Signature} & \texttt{public \texttt{ResponseEntity} handleIllegalArgumentException(\texttt{IllegalArgumentException} e)} \\
        \hline
        \emph{Behaviour} & Handles  \texttt{IllegalArgumentException} and returns a bad request response with the exception message.    \\
        \hline
        \emph{Returns} & the response entity with the exception message  \\
        \hline
        \emph{Parameters} & {\begin{tabularx}{1.0\linewidth}{@{}l R@{}}
          \texttt{e}: & the IllegalArgumentException  \\
  
        \end{tabularx}} \\
        \hline
  
      \end{tabularx}}
    \end{adjustwidth}\subsubsection{Method getEvents\label{edu.kit.hci.soli.controller.EventFeedController@getEvents(java.time.LocalDateTime,java.time.LocalDateTime,java.lang.Long,edu.kit.hci.soli.config.security.SoliUserDetails)}}
    \begin{adjustwidth}{1.5em}{0pt}
      {\begin{tabularx}{1.0\linewidth}{@{}l R@{}}
        \emph{Signature} & \texttt{public \texttt{List} getEvents(\texttt{LocalDateTime} start, \texttt{LocalDateTime} end, \texttt{Long} roomId, \texttt{\hyperref[edu.kit.hci.soli.config.security.SoliUserDetails]{\texttt{SoliUserDetails}}} principal)} \\
        \hline
        \emph{Behaviour} & Retrieves calendar events within the specified time range.    \\
        \hline
        \emph{Returns} & the list of calendar events  \\
        \hline
        \emph{Parameters} & {\begin{tabularx}{1.0\linewidth}{@{}l R@{}}
          \texttt{start}: &     the start date and time  \\
          \texttt{end}: &       the end date and time  \\
          \texttt{roomId}: &    the id of the room to show  \\
          \texttt{principal}: & the authenticated user details  \\
  
        \end{tabularx}} \\
        \hline
  
      \end{tabularx}}
    \end{adjustwidth}
  \end{adjustwidth}\subsection{Class LoginController\label{edu.kit.hci.soli.controller.LoginController} }
  \begin{adjustwidth}{1.5em}{0pt}
    {\begin{tabularx}{1.0\linewidth}{@{}l R@{}}
      \emph{Signature} & \texttt{public  class \texttt{\hyperref[edu.kit.hci.soli.controller.LoginController]{\texttt{LoginController}}}} \\
      \hline
      \emph{Behaviour} & Controller for handling login-related requests.  \\
      \hline
  
    \end{tabularx}}\subsubsection{Constructor\label{edu.kit.hci.soli.controller.LoginController@edu.kit.hci.soli.controller.LoginController(edu.kit.hci.soli.service.UserService)}}
    \begin{adjustwidth}{1.5em}{0pt}
      {\begin{tabularx}{1.0\linewidth}{@{}l R@{}}
        \emph{Signature} & \texttt{public \texttt{\hyperref[edu.kit.hci.soli.controller.LoginController]{\texttt{LoginController}}}(\texttt{\hyperref[edu.kit.hci.soli.service.UserService]{\texttt{UserService}}} userService)} \\
        \hline
        \emph{Behaviour} & Constructs a LoginController with the specified  \texttt{\hyperref[edu.kit.hci.soli.service.UserService]{\texttt{UserService}}}.    \\
        \hline
        \emph{Parameters} & {\begin{tabularx}{1.0\linewidth}{@{}l R@{}}
          \texttt{userService}: & the service for managing users  \\
  
        \end{tabularx}} \\
        \hline
  
      \end{tabularx}}
    \end{adjustwidth}\subsubsection{Method loginGuest\label{edu.kit.hci.soli.controller.LoginController@loginGuest(org.springframework.ui.Model)}}
    \begin{adjustwidth}{1.5em}{0pt}
      {\begin{tabularx}{1.0\linewidth}{@{}l R@{}}
        \emph{Signature} & \texttt{public \texttt{String} loginGuest(\texttt{Model} model)} \\
        \hline
        \emph{Behaviour} & Displays the login UI for guests.    \\
        \hline
        \emph{Returns} & the view name  \\
        \hline
        \emph{Parameters} & {\begin{tabularx}{1.0\linewidth}{@{}l R@{}}
          \texttt{model}: & the model to be used in the view  \\
  
        \end{tabularx}} \\
        \hline
  
      \end{tabularx}}
    \end{adjustwidth}\subsubsection{Method login\label{edu.kit.hci.soli.controller.LoginController@login(jakarta.servlet.http.HttpServletRequest,org.springframework.ui.Model)}}
    \begin{adjustwidth}{1.5em}{0pt}
      {\begin{tabularx}{1.0\linewidth}{@{}l R@{}}
        \emph{Signature} & \texttt{public \texttt{String} login(\texttt{HttpServletRequest} request, \texttt{Model} model)} \\
        \hline
        \emph{Behaviour} & Displays the login UI.    \\
        \hline
        \emph{Returns} & the view name  \\
        \hline
        \emph{Parameters} & {\begin{tabularx}{1.0\linewidth}{@{}l R@{}}
          \texttt{request}: & the HTTP request  \\
          \texttt{model}: &   the model to be used in the view  \\
  
        \end{tabularx}} \\
        \hline
  
      \end{tabularx}}
    \end{adjustwidth}
  \end{adjustwidth}\subsection{Class LoginControllerAdvice\label{edu.kit.hci.soli.controller.LoginControllerAdvice} }
  \begin{adjustwidth}{1.5em}{0pt}
    {\begin{tabularx}{1.0\linewidth}{@{}l R@{}}
      \emph{Signature} & \texttt{public  class \texttt{\hyperref[edu.kit.hci.soli.controller.LoginControllerAdvice]{\texttt{LoginControllerAdvice}}}} \\
      \hline
      \emph{Behaviour} & Controller advice for injecting the login state.  \\
      \hline
  
    \end{tabularx}}\subsubsection{Constructor\label{edu.kit.hci.soli.controller.LoginControllerAdvice@edu.kit.hci.soli.controller.LoginControllerAdvice(edu.kit.hci.soli.service.UserService)}}
    \begin{adjustwidth}{1.5em}{0pt}
      {\begin{tabularx}{1.0\linewidth}{@{}l R@{}}
        \emph{Signature} & \texttt{public \texttt{\hyperref[edu.kit.hci.soli.controller.LoginControllerAdvice]{\texttt{LoginControllerAdvice}}}(\texttt{\hyperref[edu.kit.hci.soli.service.UserService]{\texttt{UserService}}} userService)} \\
        \hline
        \emph{Behaviour} & Constructs a LoginControllerAdvice with the specified UserService.    \\
        \hline
        \emph{Parameters} & {\begin{tabularx}{1.0\linewidth}{@{}l R@{}}
          \texttt{userService}: & the service for managing users  \\
  
        \end{tabularx}} \\
        \hline
  
      \end{tabularx}}
    \end{adjustwidth}\subsubsection{Method getLoginStateModel\label{edu.kit.hci.soli.controller.LoginControllerAdvice@getLoginStateModel(edu.kit.hci.soli.config.security.SoliUserDetails,org.springframework.security.web.csrf.CsrfToken)}}
    \begin{adjustwidth}{1.5em}{0pt}
      {\begin{tabularx}{1.0\linewidth}{@{}l R@{}}
        \emph{Signature} & \texttt{public \texttt{\hyperref[edu.kit.hci.soli.dto.LoginStateModel]{\texttt{LoginStateModel}}} getLoginStateModel(\texttt{\hyperref[edu.kit.hci.soli.config.security.SoliUserDetails]{\texttt{SoliUserDetails}}} principal, \texttt{CsrfToken} csrf)} \\
        \hline
        \emph{Behaviour} & Retrieves the login state model based on the authenticated user details.    \\
        \hline
        \emph{Returns} & the login state model  \\
        \hline
        \emph{Parameters} & {\begin{tabularx}{1.0\linewidth}{@{}l R@{}}
          \texttt{principal}: & the authenticated user details  \\
          \texttt{csrf}: &      the CSRF token  \\
  
        \end{tabularx}} \\
        \hline
  
      \end{tabularx}}
    \end{adjustwidth}\subsubsection{Method getCsrfToken\label{edu.kit.hci.soli.controller.LoginControllerAdvice@getCsrfToken(jakarta.servlet.http.HttpServletRequest)}}
    \begin{adjustwidth}{1.5em}{0pt}
      {\begin{tabularx}{1.0\linewidth}{@{}l R@{}}
        \emph{Signature} & \texttt{public \texttt{CsrfToken} getCsrfToken(\texttt{HttpServletRequest} request)} \\
        \hline
        \emph{Behaviour} & Retrieves the CSRF token from the request.    \\
        \hline
        \emph{Returns} & the CSRF token  \\
        \hline
        \emph{Parameters} & {\begin{tabularx}{1.0\linewidth}{@{}l R@{}}
          \texttt{request}: & the HTTP request  \\
  
        \end{tabularx}} \\
        \hline
  
      \end{tabularx}}
    \end{adjustwidth}
  \end{adjustwidth}\subsection{Class MainController\label{edu.kit.hci.soli.controller.MainController} }
  \begin{adjustwidth}{1.5em}{0pt}
    {\begin{tabularx}{1.0\linewidth}{@{}l R@{}}
      \emph{Signature} & \texttt{public  class \texttt{\hyperref[edu.kit.hci.soli.controller.MainController]{\texttt{MainController}} extends \texttt{AbstractErrorController}}} \\
      \hline
      \emph{Behaviour} & Main controller for miscellaneous status requests.  \\
      \hline
  
    \end{tabularx}}\subsubsection{Constructor\label{edu.kit.hci.soli.controller.MainController@edu.kit.hci.soli.controller.MainController(org.springframework.boot.web.servlet.error.DefaultErrorAttributes)}}
    \begin{adjustwidth}{1.5em}{0pt}
      {\begin{tabularx}{1.0\linewidth}{@{}l R@{}}
        \emph{Signature} & \texttt{public \texttt{\hyperref[edu.kit.hci.soli.controller.MainController]{\texttt{MainController}}}(\texttt{DefaultErrorAttributes} errorAttributes)} \\
        \hline
        \emph{Behaviour} & Constructs an MainController with the specified  DefaultErrorAttributes .    \\
        \hline
        \emph{Parameters} & {\begin{tabularx}{1.0\linewidth}{@{}l R@{}}
          \texttt{errorAttributes}: & the default error attributes  \\
  
        \end{tabularx}} \\
        \hline
  
      \end{tabularx}}
    \end{adjustwidth}\subsubsection{Method getDisabled\label{edu.kit.hci.soli.controller.MainController@getDisabled(org.springframework.ui.Model)}}
    \begin{adjustwidth}{1.5em}{0pt}
      {\begin{tabularx}{1.0\linewidth}{@{}l R@{}}
        \emph{Signature} & \texttt{public \texttt{String} getDisabled(\texttt{Model} model)} \\
        \hline
        \emph{Behaviour} & Returns the view for disabled users.    \\
        \hline
        \emph{Returns} & the view name  \\
        \hline
        \emph{Parameters} & {\begin{tabularx}{1.0\linewidth}{@{}l R@{}}
          \texttt{model}: & the model to be used in the view  \\
  
        \end{tabularx}} \\
        \hline
  
      \end{tabularx}}
    \end{adjustwidth}\subsubsection{Method handleError\label{edu.kit.hci.soli.controller.MainController@handleError(org.springframework.ui.Model,jakarta.servlet.http.HttpServletRequest)}}
    \begin{adjustwidth}{1.5em}{0pt}
      {\begin{tabularx}{1.0\linewidth}{@{}l R@{}}
        \emph{Signature} & \texttt{public \texttt{String} handleError(\texttt{Model} model, \texttt{HttpServletRequest} request)} \\
        \hline
        \emph{Behaviour} & Handles errors and returns the appropriate error view.    \\
        \hline
        \emph{Returns} & the view name  \\
        \hline
        \emph{Parameters} & {\begin{tabularx}{1.0\linewidth}{@{}l R@{}}
          \texttt{model}: &   the model to be used in the view  \\
          \texttt{request}: & the HTTP request  \\
  
        \end{tabularx}} \\
        \hline
  
      \end{tabularx}}
    \end{adjustwidth}
  \end{adjustwidth}\subsection{Class UsersController\label{edu.kit.hci.soli.controller.UsersController} }
  \begin{adjustwidth}{1.5em}{0pt}
    {\begin{tabularx}{1.0\linewidth}{@{}l R@{}}
      \emph{Signature} & \texttt{public  class \texttt{\hyperref[edu.kit.hci.soli.controller.UsersController]{\texttt{UsersController}}}} \\
      \hline
      \emph{Behaviour} & Controller class for managing user-related operations.  \\
      \hline
  
    \end{tabularx}}\subsubsection{Constructor\label{edu.kit.hci.soli.controller.UsersController@edu.kit.hci.soli.controller.UsersController(edu.kit.hci.soli.service.UserService)}}
    \begin{adjustwidth}{1.5em}{0pt}
      {\begin{tabularx}{1.0\linewidth}{@{}l R@{}}
        \emph{Signature} & \texttt{public \texttt{\hyperref[edu.kit.hci.soli.controller.UsersController]{\texttt{UsersController}}}(\texttt{\hyperref[edu.kit.hci.soli.service.UserService]{\texttt{UserService}}} userService)} \\
        \hline
        \emph{Behaviour} & Constructs a UsersController with the specified  \texttt{\hyperref[edu.kit.hci.soli.service.UserService]{\texttt{UserService}}}.    \\
        \hline
        \emph{Parameters} & {\begin{tabularx}{1.0\linewidth}{@{}l R@{}}
          \texttt{userService}: & the service for managing users  \\
  
        \end{tabularx}} \\
        \hline
  
      \end{tabularx}}
    \end{adjustwidth}\subsubsection{Method getUsers\label{edu.kit.hci.soli.controller.UsersController@getUsers(org.springframework.ui.Model,jakarta.servlet.http.HttpServletResponse,edu.kit.hci.soli.config.security.SoliUserDetails)}}
    \begin{adjustwidth}{1.5em}{0pt}
      {\begin{tabularx}{1.0\linewidth}{@{}l R@{}}
        \emph{Signature} & \texttt{public \texttt{String} getUsers(\texttt{Model} model, \texttt{HttpServletResponse} response, \texttt{\hyperref[edu.kit.hci.soli.config.security.SoliUserDetails]{\texttt{SoliUserDetails}}} principal)} \\
        \hline
        \emph{Behaviour} & Retrieves all manageable users and returns the view for the users page.    \\
        \hline
        \emph{Returns} & the view name  \\
        \hline
        \emph{Parameters} & {\begin{tabularx}{1.0\linewidth}{@{}l R@{}}
          \texttt{model}: &     the model to be used in the view  \\
          \texttt{response}: &  the HTTP response  \\
          \texttt{principal}: & the authenticated user details  \\
  
        \end{tabularx}} \\
        \hline
  
      \end{tabularx}}
    \end{adjustwidth}\subsubsection{Method deactivateUser\label{edu.kit.hci.soli.controller.UsersController@deactivateUser(org.springframework.ui.Model,jakarta.servlet.http.HttpServletResponse,edu.kit.hci.soli.config.security.SoliUserDetails,java.lang.Long)}}
    \begin{adjustwidth}{1.5em}{0pt}
      {\begin{tabularx}{1.0\linewidth}{@{}l R@{}}
        \emph{Signature} & \texttt{public \texttt{String} deactivateUser(\texttt{Model} model, \texttt{HttpServletResponse} response, \texttt{\hyperref[edu.kit.hci.soli.config.security.SoliUserDetails]{\texttt{SoliUserDetails}}} principal, \texttt{Long} userId)} \\
        \hline
        \emph{Behaviour} & Deactivates a user by their ID.    \\
        \hline
        \emph{Returns} & the view name  \\
        \hline
        \emph{Parameters} & {\begin{tabularx}{1.0\linewidth}{@{}l R@{}}
          \texttt{model}: &     the model to be used in the view  \\
          \texttt{response}: &  the HTTP response  \\
          \texttt{principal}: & the authenticated user details  \\
          \texttt{userId}: &    the ID of the user to be deactivated  \\
  
        \end{tabularx}} \\
        \hline
  
      \end{tabularx}}
    \end{adjustwidth}
  \end{adjustwidth}
\end{adjustwidth}