%!TEX root = ../main.tex

\chapter{Controller}
\label{ch:controller}

Das Controller-Layer der Anwendung ist für die Verarbeitung von Anfragen und die Erzeugung von Antworten zuständig.
Es dekodiert und prüft dabei in \gls{HTTP}-Anfragen enthaltene Eingaben und generiert Parameter für die \gls{JTE}-Templates der View.
Zur Bearbeitung der überprüften Anfragen werden die Services des Service-Layers genutzt, welche die Datenverarbeitung kapseln.
Da einige Endpunkte eine Anmeldung benötigen, werden unangemeldete Nutzende, welche versuchen, mit einem solchen Endpunkt zu interagieren, automatisch auf die Anmeldeseite weitergeleitet.
Das Controller-Layer bietet zur Interaktion durch Nutzer die folgenden Endpunkte, welche die \gls{HTTP}-Oberfläche der Anwendung kapseln und die im Pflichtenheft beschriebenen Ansichten modellieren.


Für die Ansicht \textit{Kalender} (siehe \hyperref[edu.kit.hci.soli.controller.CalendarController]{\texttt{CalendarController}}) werden folgende Endpunkte bereitgestellt:
\begin{itemize}
    \item \texttt{GET /} gibt alle Buchungen für den Standardraum mit der fixierten ID \texttt{1} zurück.
    \item \texttt{GET /\{roomId\}} gibt die Kalenderansicht für den Raum mit der ID \texttt{roomId} zurück.
          Als Implementation für den Kalender wird dabei \gls{FullCalendar} genutzt.
    \item \texttt{GET /api/\{roomId\}/events} gibt alle Buchungen für einen Raum im \gls{FullCalendar}-Format zurück.
          Dieser Endpunkt wird von \gls{FullCalendar} genutzt, um die Buchungen im Kalender anzuzeigen.
          Links zur Ansicht \texttt{Termin} werden für die Termine eingebettet.
          Da es sich bei diesem Endpunkt um einen \gls{REST}-Endpunkt handelt, ist er separat in \hyperref[edu.kit.hci.soli.controller.EventFeedController]{\texttt{EventFeedController}} implementiert.
\end{itemize}

Für die Ansichten \textit{Termin} und \textit{Terminübersicht} (siehe \hyperref[edu.kit.hci.soli.controller.BookingViewController]{\texttt{BookingViewController}}) werden folgende Endpunkte bereitgestellt:
\begin{itemize}
    \item \texttt{GET /\{roomId\}/bookings} gibt die Terminübersicht des angemeldeten Nutzers für den Raum mit der ID \texttt{roomId} zurück.
    \item \texttt{GET /\{roomId\}/bookings/\{eventId\}} gibt die Ansicht \textit{Termin} für den Termin \texttt{eventId} zurück.
          Falls ein Administrator oder die Person, welche den Termin erstellte, angemeldet ist, wird eine Option, den Termin zu löschen, als Link eingebettet.
          Falls der Termin als kollaborativ markiert ist, wird ein link zur Ansicht \textit{Termin-Erstellen} mit den Start- und Endzeiten des angewählten Termins eingebettet.
    \item \texttt{DELETE /\{roomId\}/bookings/\{eventId\}} löscht den Termin \texttt{eventId} und leitet Nutzende auf die Terminübersicht weiter.
\end{itemize}

Für die Ansicht \textit{Termin-Erstellen} (siehe \hyperref[edu.kit.hci.soli.controller.BookingCreateController]{\texttt{BookingCreateController}}) werden folgende Endpunkte bereitgestellt:
\begin{itemize}
    \item \texttt{GET /\{roomId\}/bookings/new} gibt das Formular (implementiert als \gls{HTML-Form}) zur Erstellung eines neuen Termins für den Raum mit der ID \texttt{roomId} zurück.
    \item \texttt{POST /\{roomId\}/bookings/new} erstellt einen neuen Termin für den Raum mit der ID \texttt{roomId} und leitet Nutzende auf die Ansicht \textit{Termin} weiter.
          Die Form-Eingaben des Formulars werden dabei dekodiert und geprüft.
          Falls ein Konflikt mit einem bereits bestehenden Termin besteht, werden Nutzende auf die Konfliktseite weitergeleitet.
          In diesem Fall wird der dekodierte Termin als \hyperref[edu.kit.hci.soli.domain.Booking]{\texttt{Booking}} in der Session gespeichert.
    \item \texttt{POST /\{roomId\}/bookings/new/conflict} bietet Nutzenden die Möglichkeit, einen Konflikt entsprechend der Spezifikation im Pflichtenheft zu lösen.
          Dabei wird eine mögliche Lösung vorgeschlagen, welche Nutzende bestätigen oder ablehnen können.
          Im Falle der Bestätigung wird der in der Session gespeicherte Termin erstellt und Nutzende auf die Ansicht \textit{Termin} weitergeleitet.
          War unter den Konflikten ein Termin, welcher Zustimmung zur Kollaboration benötigt, wird entsprechend eine E-Mail an die betroffenen Nutzenden versendet.
          War unter den Konflikten ein Termin, welcher durch die Erstellung gelöscht wurde, wird entsprechend eine E-Mail an die betroffenen Nutzenden versendet.
\end{itemize}

Zum akzeptieren von Kollaborationen (siehe \hyperref[edu.kit.hci.soli.controller.CollaborationRequestController]{\texttt{CollaborationRequestController}}) werden folgende Endpunkte bereitgestellt:
\begin{itemize}
    \item \texttt{GET /\{roomId\}/bookings/\{eventId\}/collaboration} gibt das Formular zur Bestätigung einer Kollaboration für den Termin \texttt{eventId} zurück.
    \item \texttt{PUT /\{roomId\}/bookings/\{eventId\}/collaboration} akzeptiert die Kollaboration für den Termin \texttt{eventId} und leitet Nutzende auf die Ansicht \textit{Termin} weiter.
    \item \texttt{DELETE /\{roomId\}/bookings/\{eventId\}/collaboration} lehnt die Kollaboration für den Termin \texttt{eventId} ab und leitet Nutzende auf die Ansicht \textit{Terminübersicht} weiter.
\end{itemize}

Für die Ansicht \textit{Login} (siehe \hyperref[edu.kit.hci.soli.controller.LoginController]{\texttt{LoginController}}) werden folgende Endpunkte bereitgestellt:
\begin{itemize}
    \item \texttt{GET /login} gibt das Formular zur Anmeldung zurück. Die Anmeldung kann mit einem eingebetteten Formular als Administrator oder mit einem Link als Gast oder über \gls{OIDC} erfolgen.
          Für die Anmeldung als Administrator wird eine \gls{HTML-Form} genutzt, welche die Eingabe eines Passworts fordert.
    \item \texttt{POST /login} erlaubt die Anmeldung mit lokalen Anmeldedaten (also als Gast oder Administrator) und leitet Nutzende auf die Ansicht, welche die Anmeldung verursachte, oder die Ansicht \textit{Kalender} weiter.
          Dieser Endpunkt wird bei Bestätigung einer Gast- oder Administratoren-Anmeldung aufgerufen.
    \item \texttt{GET /login/guest} gibt das Formular zur Anmeldung als Gast zurück.
    \item \texttt{GET /oauth2/authorization/kit} verarbeitet die Anmeldung über \gls{OIDC} und leitet Nutzende auf die Ansicht, welche die Anmeldung verursachte, oder die Ansicht \textit{Kalender} weiter.
          Auf diesen Endpunkt werden Nutzende bei Bestätigung einer \gls{OIDC}-Anmeldung durch den OIDC-Server weitergeleitet.
\end{itemize}

Für die Ansicht \textit{Kontoliste} (siehe \hyperref[edu.kit.hci.soli.controller.UsersController]{\texttt{UsersController}}) werden folgende Endpunkte bereitgestellt:
\begin{itemize}
    \item \texttt{GET /admin/users} gibt die Kontoliste aller Nutzenden zurück. Das Administrationskonto wird dabei nicht angezeigt.
          Für jedes Konto wird ein Link zur Deaktivierung als \gls{HTML-Form} eingebettet (dies erlaubt die Nutzung von \texttt{PUT}-Anfragen).
    \item \texttt{GET /admin/users/{userId}/deactivate} gibt das Formular zur Deaktivierung des Kontos mit der ID \texttt{userId} zurück.
    \item \texttt{PUT /admin/users/{userId}/deactivate} deaktiviert das Konto mit der ID \texttt{userId} und leitet Nutzende auf die Kontoliste weiter.
    \item \texttt{PUT /admin/users/{userId}/reactivate} reaktiviert das Konto mit der ID \texttt{userId} und leitet Nutzende auf die Kontoliste weiter.
    \item \texttt{GET /admin/users/disable-guests} gibt das Formular zur Deaktivierung der Anmeldung als Gast zurück.
    \item \texttt{PUT /admin/users/disable-guests} deaktiviert den Login durch Gastkonten und leitet Nutzende auf die Kontoliste weiter.
          Zudem werden alle Buchungen von Gastkonten gelöscht.
    \item \texttt{PUT /admin/users/enable-guests} reaktiviert den Login durch Gastkonten und leitet Nutzende auf die Kontoliste weiter.
\end{itemize}

Außerdem werden folgende allgemeine Endpunkte (siehe \hyperref[edu.kit.hci.soli.controller.MainController]{\texttt{MainController}}) bereitgestellt:
\begin{itemize}
    \item \texttt{GET /error} gibt eine Fehlerseite zurück. Dieser Endpunkt wird von Spring automatisch aufgerufen, wenn ein Fehler auftritt.
    \item \texttt{GET /disabled} gibt eine Seite zurück, die Nutzende darüber informiert, dass ihr Konto deaktiviert wurde.
\end{itemize}

\InputIfFileExists{javadoc/edu.kit.hci.soli.controller}{}{}