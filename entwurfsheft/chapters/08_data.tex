%!TEX root = ../main.tex

\chapter{Daten}
\label{ch:data}

\todo{Den Daten layer beschreiben (die Repos)}

Der Daten Layer befindet sich im Paket \textit{repository} dieses enthält Schnittstellen, die für den Datenzugriff und die Verwaltung von Entitäten in der Datenbank verantwortlich sind. 
Es verwendet das Spring Data JPA Framework, um \gls{CRUD}-Operationen und benutzerdefinierte Abfragen auf den Entitäten durchzuführen.  
Hauptaufgaben des Daten Layers sind:

\begin{itemize}
    \item Verwaltung von Entitäten: Es definiert Schnittstellen, die von JpaRepository erben, um CRUD-Operationen auf den Entitäten durchzuführen.
    \item Benutzerdefinierte Abfragen: Es definiert benutzerdefinierte Abfragen, um spezifische Daten aus der Datenbank zu extrahieren.
\end{itemize}

Welche Operation auf welcher Entität durchgeführt werden kann, ist im folgenden dargestellt: 

\InputIfFileExists{javadoc/edu.kit.hci.soli.repository}{}{}
