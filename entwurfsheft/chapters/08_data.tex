%!TEX root = ../main.tex

\chapter{Daten}
\label{ch:data}

Der Daten-Layer befindet sich im Paket \hyperref[edu.kit.hci.soli.repository]{\texttt{repository}} dieses enthält Schnittstellen, die für den Datenzugriff und die Verwaltung von Entitäten in der Datenbank verantwortlich sind.
Hierfür wird das Spring Data JPA Framework verwendet, um \gls{CRUD}-Operationen und benutzerdefinierte Abfragen auf den Entitäten durchzuführen.
Die Hauptaufgaben des Daten-Layers sind:

\begin{itemize}
    \item Verwaltung von Entitäten: Es definiert Schnittstellen, die von JPA-Repository erben, um CRUD-Operationen auf den Entitäten durchzuführen.
    \item Benutzerdefinierte Abfragen: Es definiert benutzerdefinierte Abfragen, um spezifische Daten aus der Datenbank zu extrahieren.
\end{itemize}

Welche Operationen auf welcher Entität durchgeführt werden können, ist im Folgenden dargestellt.

\InputIfFileExists{javadoc/edu.kit.hci.soli.repository}{}{}
