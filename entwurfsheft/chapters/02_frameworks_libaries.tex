\chapter{Frameworks und Libaries}
\label{ch:frameworks_libaries}

Für das Produkt \textit{Soli} werden unterschiedliche Frameworks und Libraries verwendet, um die Entwicklung zu erleichtern und die Qualität des Codes zu erhöhen. In diesem Kapitel werden die wichtigsten dieser Tools vorgestellt.
Im Backend wird das Spring Framework verwendet. Dabei kommen die Module: Spring Boot JPA, PostgreSQL, Flyway, Spring Boot Mail, Spring Boot OAuth2 Client, Spring Boot Web und Spring Boot Security zum Einsatz. Außerdem werden statische \gls{HTML} Seiten mit der Java Template Engine (JTE) generiert. Im Frontend wird das \gls{CSS} Framework und die JavaScript Bibliothek \gls{FullCalendar} verwendet.
Zum Testen der Anwendung wird JUnit, Mockito und Testcontainers verwendet.
Für die Build-Automatisierung wird Gradle verwendet.
Außerdem wird Docker verwendet, um die Anwendung in einem Container zu betreiben.